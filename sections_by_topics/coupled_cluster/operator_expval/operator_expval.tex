\documentclass[a4paper,11pt]{article}
\usepackage{braket}
\usepackage{physics}
\usepackage{parskip}
\usepackage{bm}
\usepackage{float}
\usepackage[utf8]{inputenc}
\usepackage{amsmath}
\usepackage{pgfplots}
\usepackage{mathrsfs}
\usepackage{simpler-wick}
\usepackage{enumerate}
\usepackage{csquotes}
\usepackage{subcaption}
\usepackage{fancyhdr}
\usepackage{tablefootnote}
\usepackage{microtype}
\usepackage{cleveref}
\usepackage{titling}
\usepackage{erewhon}
\usepackage[a4paper,width=150mm,top=25mm,bottom=25mm]{geometry}
\usepackage[Sonny]{fncychap}
\usepackage[calcwidth]{titlesec}

\usetikzlibrary{shapes.geometric, arrows}

\allowdisplaybreaks

\pgfplotsset{compat=1.7}

%fncychap layout (for chapter page)
%\renewcommand{\thechapter}{\Roman{chapter}}
\ChNameVar{\bfseries\LARGE\fontfamily{phv}}
\ChNumVar{\fontsize{50}{52}\usefont{OT1}{ptm}{m}{n}\selectfont}
\ChTitleVar{\LARGE\rm\bfseries}
\ChRuleWidth{0.8pt}

%fancyhdr layout (for header and footer)
\pagestyle{fancy}
\fancyhead{}
\fancyhead[C]{\fontsize{9}{9}\itshape{\rightmark}}
\setlength{\headheight}{15pt}

\begin{document}


\section{Expected Value of Operators}
    For a general operator $\hat{O}$, the expectation value is:
    \begin{equation}
        \bar{O} = \langle \hat{O} \rangle = \frac{\langle \Psi|\hat{O}|\Psi\rangle}{\langle \Psi | \Psi \rangle}
    \end{equation}
    It could be written in the normal product form:
    \begin{align}
        \hat{O} =& \sum_{pq}\langle p|\hat{o}|q\rangle \hat{p}^{\dagger}\hat{q} \\ \nonumber
                =& \sum_{pq} o _{pq}(\{\hat{p}^{\dagger}\hat{q}\} + \{\wick[sep=0.35em]{\c1{\hat{p}^{\dagger}}\c1{\hat{q}}}\}) \\ \nonumber
                =& \sum_{pq} o _{pq}\{\hat{p}^{\dagger}\hat{q}\} + \sum_{i}o _{ii}
    \end{align}
    since
    \begin{equation}
        \{\wick[sep=0.35em]{\c1{\hat{p}^{\dagger}}\c1{\hat{q}}}\} = \delta pq \delta _{p \in \text{occ}}
    \end{equation}
    Then normal one-body operator expectation value is then:
    \begin{equation}
        \bar{O}_{\text{N}} = \frac{\langle 0|e ^{\hat{T}^{\dagger}}\hat{O}_{\text{N}}e ^{\hat{T}^{\dagger}}|0\rangle}{\langle 0|e ^{\hat{T}^{\dagger}}e ^{\hat{T}}|0 \rangle}
    \end{equation}
    From \textit{Bartlett \& Shavitt \S 11.1}, we have (valid for all normal-ordered operators):
    \begin{equation}
        \langle 0|e ^{\hat{T}^{\dagger}}\hat{O}_{\text{N}}e ^{\hat{T}}|0 \rangle = \langle 0|e ^{\hat{T}^{\dagger}}e ^{\hat{T}}|0 \rangle \langle 0|e ^{\hat{T}^{\dagger}}\hat{O}_{\text{N}}e ^{\hat{T}}|0 \rangle _{\text{C}}
    \end{equation}
    and therefore,
    \begin{equation}
        \bar{O}_{\text{N}} = \langle 0|e ^{\hat{T}^{\dagger}}\hat{O}_{\text{N}}e ^{\hat{T}}|0 \rangle _{\text{C}}
    \end{equation}
    We can use this result to obtain the correlation energy for CC, as:
    \begin{equation}
        E _{\text{corr}} = \langle \hat{H}_{\text{N}}\rangle = \frac{\langle 0|e ^{\hat{T}^{\dagger}}\hat{H}_{\text{N}}e ^{\hat{T}}|0 \rangle}{\langle 0|e ^{\hat{T}^{\dagger}}e ^{\hat{T}}|0 \rangle} = \langle 0|e ^{\hat{T}^{\dagger}}\hat{H}_{\text{N}}e ^{\hat{T}}|0 \rangle _{\text{C}}
    \end{equation}
    The more familiar form of CC correlation energy is:
    \begin{equation}
        E _{\text{corr}} = \langle 0|\hat{H}_{\text{N}}e ^{\hat{T}}|0 \rangle _{\text{C}} = \langle 0|e ^{-\hat{T}}\hat{H}_{\text{N}}e ^{\hat{T}}|0 \rangle
    \end{equation}
    These two forms are actually equivalent, demonstrated with the aid of the fact that $e ^{\hat{T}(\hat{P}+\hat{Q})}e ^{-\hat{T}} = \hat{1}$, where $\hat{P} = |0\rangle \langle 0|$ and $Q = \hat{1} - \hat{P} = \sum_{i \neq 0}|i \rangle \langle i|$, as following:
    \begin{align}
        E _{\text{corr}} =& \frac{\langle 0|e ^{\hat{T}^{\dagger}}\hat{H}_{\text{N}}e ^{\hat{T}}|0 \rangle}{\langle 0|e ^{\hat{T}^{\dagger}}e ^{\hat{T}}|0 \rangle}  \nonumber \\
         =& \frac{\langle 0|e ^{\hat{T}^{\dagger}}e ^{\hat{T}}(\hat{P}+\hat{Q})e ^{-\hat{T}} \hat{H}_{\text{N}}e ^{\hat{T}}|0 \rangle}{\langle 0|e ^{\hat{T}^{\dagger}}e ^{\hat{T}}|0 \rangle}  \\ \nonumber
         =& \frac{\langle 0|e ^{\hat{T}^{\dagger}}e ^{\hat{T}}\hat{P}e ^{-\hat{T}} \hat{H}_{\text{N}}e ^{\hat{T}}|0 \rangle}{\langle 0|e ^{\hat{T}^{\dagger}}e ^{\hat{T}}|0 \rangle} +\frac{\langle 0|e ^{\hat{T}^{\dagger}}e ^{\hat{T}}\hat{Q}e ^{-\hat{T}} \hat{H}_{\text{N}}e ^{\hat{T}}|0 \rangle}{\langle 0|e ^{\hat{T}^{\dagger}}e ^{\hat{T}}|0 \rangle}  \\ \nonumber
         =& \frac{\langle 0|e ^{\hat{T}^{\dagger}}e ^{\hat{T}}|0 \rangle \langle 0|e ^{-\hat{T}}\hat{H}_{\text{N}}e ^{\hat{T}}|0 \rangle}{\langle 0|e ^{\hat{T}^{\dagger}}e ^{\hat{T}}|0 \rangle} \\ \nonumber
         =& \langle 0|e ^{-\hat{T}}\hat{H}_{\text{N}}e ^{\hat{T}}|0 \rangle \\ \nonumber
         =& \langle 0|\hat{H}_{\text{N}}e ^{\hat{T}}|0 \rangle _{\text{C}}
    \end{align}  
    in which we used the fact that:
    \begin{equation}
        \hat{Q}e ^{-\hat{T}} \hat{H}_{\text{N}}e ^{\hat{T}} |0\rangle = \hat{Q}\hat{\mathcal{H}}|0\rangle = 0
    \end{equation}
    which is equivalently:
    \begin{equation}
        \langle \Phi _{ij \dots}^{ab \dots} | \hat{\mathcal{H}} |0\rangle = 0
    \end{equation}
    which are the CC amplitudes equations.

\end{document}