\documentclass[a4paper,11pt]{article}
\usepackage{braket}
\usepackage{physics}
\usepackage{parskip}
\usepackage{bm}
\usepackage{float}
\usepackage[utf8]{inputenc}
\usepackage{amsmath}
\usepackage{pgfplots}
\usepackage{mathrsfs}
\usepackage{simpler-wick}
\usepackage{enumerate}
\usepackage{csquotes}
\usepackage{subcaption}
\usepackage{fancyhdr}
\usepackage{tablefootnote}
\usepackage{microtype}
\usepackage{cleveref}
\usepackage{titling}
\usepackage{erewhon}
\usepackage[a4paper,width=150mm,top=25mm,bottom=25mm]{geometry}
\usepackage[Sonny]{fncychap}
\usepackage[calcwidth]{titlesec}

\usetikzlibrary{shapes.geometric, arrows}

\allowdisplaybreaks

\pgfplotsset{compat=1.7}

%fncychap layout (for chapter page)
%\renewcommand{\thechapter}{\Roman{chapter}}
\ChNameVar{\bfseries\LARGE\fontfamily{phv}}
\ChNumVar{\fontsize{50}{52}\usefont{OT1}{ptm}{m}{n}\selectfont}
\ChTitleVar{\LARGE\rm\bfseries}
\ChRuleWidth{0.8pt}

%fancyhdr layout (for header and footer)
\pagestyle{fancy}
\fancyhead{}
\fancyhead[C]{\fontsize{9}{9}\itshape{\rightmark}}
\setlength{\headheight}{15pt}

\begin{document}


\section{Stationary Lagrangian}


    The CC Lagrangian is given as:
    \begin{equation}
        \mathcal{L} _{\text{CC}} = \langle \Phi _{0}|(1+\hat{\Lambda})\mathcal{H}|\Phi _{0}\rangle + \sum_{ai}f _{ai}z _{ai} + \sum_{pq}I _{pq}(\sum_{\mu \nu}C ^{*} _{\mu p}S _{\mu \nu}C _{\nu q} - \delta _{pq})
    \end{equation}
    \textcolor{red}{(Question: do I need to include other blocks of the Fock matrix?)} \\
    in which:
    \begin{equation}
        \hat{\Lambda} = \hat{\Lambda}_{1}+\hat{\Lambda}_{2} + \dots = \sum_{ia}\lambda ^{i}_{a}\{\hat{i}^{\dagger}\hat{a}\} + \frac{1}{4}\sum_{ijab}\lambda _{ab}^{ij}\{\hat{i}^{\dagger}\hat{j}^{\dagger}\hat{b}\hat{a}\} + \dots
    \end{equation}
    The first term in the Lagrangian is used to include CC energy expression and CC amplitude constraints:
    \begin{align}
        \langle \Phi _{0}|(1+\Lambda)\mathcal{H}|\Phi _{0}\rangle =& \langle \Phi _{0}|\mathcal{H}|\Phi _{0}\rangle + \langle \Phi _{0}|\hat{\Lambda}\mathcal{H}|\Phi _{0}\rangle \nonumber \\
        =& \Delta E + \sum_{\mu}\lambda _{\mu}\langle \Phi _{\mu}|\mathcal{H}|\Phi _{0}\rangle
    \end{align} 
    We need to impose the stationary conditions, w.r.t.:
    \begin{itemize}
    \item $\lambda _{\mu}$: resulting the CC amplitude equations
    \item $t _{\mu}$: resulting the CC lambda equations
    \item $z _{ai}$: the HF condition
    \item $\kappa _{ai}$: resulting the z-vector equations
    \item $I _{pq}$: resulting the orthonomality condition 
    \end{itemize}

    \subsection{CC Amplitude Equations}
    \begin{align}
        \pdv{\mathcal{L}_{\text{CC}}}{\lambda _{\mu}} &= \pdv{\lambda _{\mu}}(\langle \Phi _{0}|\mathcal{H}|\Phi _{0}\rangle + \langle \Phi _{0}|\hat{\Lambda}\mathcal{H}|\Phi _{0}\rangle + \sum_{ia}f _{ia}z _{ia}+ \sum_{pq}I _{pq}(\sum_{\mu \nu}C ^{*}_{\mu p}S _{\mu \nu}C _{\nu q} - \delta _{pq})) \\ \nonumber
        &= \langle \Phi _{\mu}|\mathcal{H}|\Phi _{0}\rangle
    \end{align}
    Hence imposing the stationary condition:
    \begin{equation}
        \pdv{\mathcal{L} _{\text{CC}}}{\lambda _{\mu}} = 0
    \end{equation}
    we get:
    \begin{equation}
        \langle \Phi _{\mu}|\mathcal{H}|\Phi _{0}\rangle = 0
    \end{equation}
    which are the CC amplitude equations. The alternative form is:
    \begin{equation}
        \hat{Q}\mathcal{H}\hat{P} = 0
    \end{equation}

    \subsection{CC Lambda Equations}
    By:
    \begin{align}
        \mathcal{H} &= (\hat{H}_{\text{N}}e ^{\hat{T}})_{\text{C}} \\ \nonumber
        &= (\hat{H}_{\text{N}}+\hat{H}_{\text{N}}\hat{T}+\frac{1}{2}\hat{H}_{\text{N}}\hat{T}^{2}+\frac{1}{6}\hat{H}_{\text{N}}\hat{T}^{3}+\frac{1}{24}\hat{H}_{\text{N}}\hat{T}^{4})_{\text{C}}
    \end{align}
    and:
    \begin{equation}
        \mathcal{L}_{\text{CC}} = \langle \Phi _{0}|(1+\hat{\Lambda})\mathcal{H}|\Phi _{0}\rangle + \sum_{p>q}f _{pq}z _{pq}
    \end{equation}
    We can find the partial derivative $\pdv*{\mathcal{L} _{\text{CC}}}{t _{\mu}}$. First let's consider (CCSD): \\
    \textcolor{red}{(Should I truncate to four-fold then take derivatives, or the other way around? Or can I show they're equivalent?)}
    \begin{align}
        \pdv{\mathcal{H}}{t _{i}^{a}} =&  \pdv{t _{i}^{a}}(\hat{H}_{\text{N}}(1+\hat{T} + \frac{1}{2}\hat{T}^{2}+ \frac{1}{6}\hat{T}^{3} + \frac{1}{24}\hat{T}^{4}+\dots))_{\text{C}} \\
        \pdv{\hat{H}_{\text{N}}}{t _{i}^{a}} =& 0 \\
        \pdv{t _{i}^{a}}(\hat{H}_{\text{N}}\hat{T}) =& \pdv{t _{i}^{a}}(\hat{H}_{\text{N}}(\hat{T}_{1}+\hat{T}_{2})) = \hat{H}_{\text{N}}\{\hat{a}^{\dagger}\hat{i}\} \\
        \pdv{t _{i}^{a}}(\frac{1}{2}\hat{H}_{\text{N}}\hat{T}^{2}) =& \frac{1}{2}\hat{H}_{\text{N}}\pdv{t _{i}^{a}}(\hat{T}_{1}^{2}+2 \hat{T}_{1}\hat{T}_{2}+\hat{T}_{2}^{2}) \\ \nonumber
        =& \frac{1}{2}\hat{H}_{\text{N}}(2 \hat{T}_{1} + 2 \hat{T}_{2})\{\hat{a}^{\dagger}\hat{i}\} \\ \nonumber
        =&\hat{H}_{\text{N}}\hat{T}\{\hat{a}^{\dagger}\hat{i}\} \\ \nonumber
        \pdv{t _{i}^{a}}(\frac{1}{6}\hat{H}_{\text{N}}\hat{T}^{3}) =& \frac{1}{6}\hat{H}_{\text{N}}\pdv{t _{i}^{a}}(\hat{T}_{1}^{3}+3 \hat{T}_{1}^{2}\hat{T}_{2} + 3 \hat{T}_{1}\hat{T}_{2}^{2} + \hat{T}_{2}^{3}) \\
        =&\frac{1}{6}\hat{H}_{\text{N}}(3 \hat{T}_{1}^{2} + 6 \hat{T}_{1}\hat{T} + 3 \hat{T}_{2}^{2})\{\hat{a}^{\dagger}\hat{i}\} \\ \nonumber
        =&\frac{1}{2}\hat{H}_{\text{N}}\hat{T}^{2}\{\hat{a}^{\dagger}\hat{i}\} \\ \nonumber
        \pdv{t _{i}^{a}}(\frac{1}{24}\hat{H}_{\text{N}}\hat{T}^{4}) =& \frac{1}{24}\hat{H}_{\text{N}}\pdv{t _{i}^{a}}(\hat{T}_{1}^{4} + 4 \hat{T}_{1}^{3}\hat{T}_{2} + 6 \hat{T}_{1}^{2}\hat{T}_{2}^{2} + 4 \hat{T}_{1}\hat{T}_{2}^{3} + \hat{T}_{2}^{4}) \\
        =& \frac{1}{24}\hat{H}_{\text{N}}(4 \hat{T}_{1}^{3} + 12 \hat{T}_{1}\hat{2}\hat{T}_{2} + 12 \hat{T}_{1}\hat{T}_{2}^{2} + 4 \hat{T}_{2}^{3})\{\hat{a}^{\dagger}\hat{i}\} \\ \nonumber
        =& \frac{1}{6}\hat{H}_{\text{N}}\hat{T}^{3}\{\hat{a}^{\dagger}\hat{i}\} \\ \nonumber
        & \dots
    \end{align}
    Therefore:
    \begin{align}
        \pdv{\mathcal{H}}{t _{i}^{a}} =& (\hat{H}_{\text{N}} (1 + \hat{T} + \frac{1}{2}\hat{T}^{2} + \frac{1}{6}\hat{T}^{3} + ...))_{\text{C}}\{\hat{a}^{\dagger}\hat{i}\} \\
        =&(\hat{H}_{\text{N}}e ^{\hat{T}})_{\text{C}}\{\hat{a}^{\dagger}\hat{i}\} \nonumber \\
        =&\mathcal{H}\{\hat{a}^{\dagger}\hat{i}\} \nonumber
    \end{align}
    \textcolor{blue}{Compare results with commutators?}
    \\
    By BCH:
    \begin{align}
        \mathcal{H} =& \hat{H}_{\text{N}} + [\hat{H}_{\text{N}},\hat{T}] + \frac{1}{2!}[[\hat{H}_{\text{N}},\hat{T}],\hat{T}] + \frac{1}{3!}[[[\hat{H}_{\text{N}},\hat{T}],\hat{T}],\hat{T}]   \nonumber \\
        &+ \frac{1}{4!}[[[[\hat{H}_{\text{N}},\hat{T}],\hat{T}],\hat{T}],\hat{T}] + \dots
    \end{align}
    \textcolor{blue}{Can't truncate yet, for the sake of taking derivatives.} \\
    Then take derivative w.r.t amplitude $t _{i}^{a}$:
    \begin{align}
        \pdv{\mathcal{H}}{t _{i}^{a}} =& \pdv{\hat{H}_{\text{N}}}{t _{i}^{a}} + \pdv{t _{i}^{a}}[\hat{H}_{\text{N}},\hat{T}] + \dots \\
        \pdv{\hat{H}_{\text{N}}}{t _{i}^{a}} =& 0 \\
        \pdv{t _{i} ^{a}}[\hat{H}_{\text{N}}, \hat{T}] =& \pdv{t _{i}^{a}}[\hat{H}_{\text{N}},\hat{T}_{\text{1}}] + 0 = [\hat{H}_{\text{N}},\{\hat{a}^{\dagger}\hat{i}\}] \\
        \pdv{t _{i}^{a}}[[\hat{H}_{\text{N}},\hat{T}],\hat{T}] =& \pdv{t _{i}^{a}}\Big([[\hat{H}_{\text{N}},\hat{T} _{1}],\hat{T}_{1}] + [[\hat{H}_{\text{N}},\hat{T}_{1}],\hat{T}_{2}] + [[\hat{H}_{\text{N}},\hat{T}_{2}],\hat{T}_{1}]+ [[\hat{H}_{\text{N}},\hat{T}_{2}],\hat{T}_{2}]\Big) \nonumber \\
        =& [[\hat{H}_{\text{N}},\pdv{\hat{T}_{1}}{t _{i}^{a}}],\hat{T}_{1}] + [[\hat{H}_{\text{N}},\hat{T}_{1}],\pdv{\hat{T}_{1}}{t _{i}^{a}}] + [[\hat{H}_{\text{N}},\pdv{\hat{T}_{1}}{t _{i}^{a}}],\hat{T}_{2}] + [[\hat{H}_{\text{N}},\hat{T}_{2}],\pdv{\hat{T}_{1}}{t _{i}^{a}}] \nonumber \\
        =& [[\hat{H}_{\text{N}},\{\hat{a}^{\dagger}\hat{i}\}],\hat{T}_{1}] + [[\hat{H}_{\text{N}},\hat{T}_{1}],\{\hat{a}^{\dagger}\hat{i}\}] + [[\hat{H}_{\text{N}},\{\hat{a}^{\dagger}\hat{i}\}],\hat{T}_{2}] + [[\hat{H}_{\text{N}},\hat{T}_{2}],\{\hat{a}^{\dagger}\hat{i}\}] \nonumber \\
        =& [[\hat{H}_{\text{N}},\{\hat{a}^{\dagger}\hat{i}\}],\hat{T}] + [[\hat{H}_{\text{N}},\hat{T}],\{\hat{a}^{\dagger}\hat{i}\}] \\ \pdv{t _{i}^{a}}[[[\hat{H}_{\text{N}},\hat{T}],\hat{T}],\hat{T}] =& \dots \nonumber \\
        &(\text{At the end of this tedious evaluation, we can find:}) \nonumber \\
        \pdv{\mathcal{H}}{t _{i}^{a}} =& [\mathcal{H}, \{\hat{a}^{\dagger}\hat{i}\}] \\
        \pdv{\mathcal{H}}{t _{ij}^{ab}} =& [\mathcal{H}, \frac{1}{4}\{\hat{a}^{\dagger}\hat{b}^{\dagger}\hat{j}\hat{i}\}]
    \end{align}
    \textcolor{blue}{I think the commutator expression may be more convenient, for the following derivation:} \\
    Now to find $\pdv*{\mathcal{L}_{\text{CC}}}{t _{\mu}}$:
    \begin{align}
        \pdv{\mathcal{L}_{\text{CC}}}{t _{i}^{a}} =& \langle \Phi _{0}|(1+\hat{\Lambda})[\mathcal{H},\{\hat{a}^{\dagger}\hat{i}\}]|\Phi _{0}\rangle  \nonumber \\
        =& \langle \Phi _{0}|(1+\hat{\Lambda})\mathcal{H}\{\hat{a}^{\dagger}\hat{i}\}|\Phi _{0}\rangle - \langle \Phi _{0}|(1+\hat{\Lambda})\{\hat{a}^{\dagger}\hat{i}\}\mathcal{H}|\Phi _{0}\rangle \nonumber \\
        =& \langle \Phi _{0}|(1+\hat{\Lambda})\mathcal{H}|\Phi _{i}^{a}\rangle - \langle \Phi _{0}|(1+\hat{\Lambda})\Delta E|\Phi _{i}^{a}\rangle \nonumber \\
        =& \langle \Phi _{0}|(1+\hat{\Lambda})(\mathcal{H}-\Delta E)|\Phi _{i}^{a}\rangle
    \end{align}
    Similarly:
    \begin{equation}
        \pdv{\mathcal{L}_{\text{CC}}}{t _{ij}^{ab}} = \langle \Phi _{0}|(1+\hat{\Lambda})(\mathcal{H} - \Delta E)|\Phi _{ij}^{ab}\rangle
    \end{equation}
    These expressions are the same as the $\Lambda$ equations obtained in previous section.

    

    \subsection{HF Condition}
    Taking the derivative of $\mathcal{L}_{\text{CC}}$ w.r.t the z-vector results in the Brillouin's condition:
    \begin{equation}
        \pdv{\mathcal{L}_{\text{CC}}}{z _{ai}} = f _{ai} = 0
    \end{equation}
    which is equivalent to the HF equation.

    \subsection{z-vector Equations}
    \textcolor{red}{Question: (related to frozen-core approximation I suppose) W.r.t what block of $\bm{\kappa}$ do I need to take derivatives for $\mathcal{L}_{\text{CC}}$?} Virtual-inactive block ($\kappa _{bj}$) is shown below because this type of orbital rotation is always non-redundant.
    \begin{align}
        \pdv{\mathcal{L}_{\text{CC}}}{\kappa _{bj}} =& \pdv{\kappa _{bj}}(\langle \Phi _{0}|(1+\hat{\Lambda})\mathcal{H}|\Phi _{0}\rangle + \sum_{ai}f _{ai}z _{ai} + \sum_{pq}I _{pq}(\sum_{\mu \nu}C ^{*}_{\mu p}S _{\mu \nu}C _{\nu q} - \delta _{pq})) \nonumber \\
        =& \pdv{\kappa _{bj}}\langle \Phi _{0}|(1+\hat{\Lambda})\mathcal{H}|\Phi _{0}\rangle + \sum_{ai}\pdv{f _{ai}}{\kappa _{bj}}z _{ai} + \sum_{pq}I _{pq}(\sum_{\mu \nu}U ^{*}_{\mu p}\pdv{\mathcal{S}_{\mu \nu}}{\kappa _{bj}}U _{\nu q})
    \end{align}
    in which:
    \begin{align}
        \mathbf{U} =& e ^{-\hat{\kappa}} \\
        \hat{\kappa} =& \sum_{p>q}\kappa _{pq}E _{pq}^{-} \\
        \bm{\mathcal{S}} =& \mathbf{C}^{\dagger}(0)\mathbf{S}\mathbf{C}(0)
    \end{align}
    To evaluate term by term, the CC energy part:
    \begin{align}
        \pdv{\kappa _{bj}}\langle \Phi _{0}(\hat{\kappa})|(1+\hat{\Lambda})\mathcal{H}|\Phi _{0}(\hat{\kappa})\rangle =& \pdv{\kappa _{bj}}\langle \Phi _{0}|e ^{\hat{\kappa}}(1+\hat{\Lambda})\mathcal{H}e ^{-\hat{\kappa}}|\Phi _{0}\rangle  \nonumber \\
        =& \pdv{\kappa _{bj}}\langle \Phi _{0}|e ^{\hat{\kappa}}\mathcal{H}e ^{-\hat{\kappa}}|\Phi _{0}\rangle + \pdv{\kappa _{bj}}\langle \Phi _{0}|e ^{\hat{\kappa}}\hat{\Lambda}\mathcal{H}e ^{-\hat{\kappa}}|\Phi _{0}\rangle
    \end{align}
    The first term (using BCH expansion):
    \begin{align}
        \pdv{\kappa _{bj}}\langle \Phi _{0}(\hat{\kappa})|\mathcal{H}|\Phi _{0}(\hat{\kappa})\rangle =& \pdv{\kappa _{bj}}\langle \Phi _{0}|\mathcal{H}+ [\mathcal{H},-\hat{\kappa}] + \frac{1}{2!}[[\mathcal{H},-\hat{\kappa}],-\hat{\kappa}] + \frac{1}{3!}[[[\mathcal{H},-\hat{\kappa}],-\hat{\kappa}],-\hat{\kappa}] + \dots|\Phi _{0}\rangle  \nonumber \\
        =&\langle \Phi _{0}|\mathcal{H} + [\mathcal{H},-E _{bj}^{-}] + [[\mathcal{H},-E _{bj}^{-}],-\hat{\kappa}] + [[\mathcal{H},-\hat{\kappa}],-E _{bj}^{-}] + \dots|\Phi _{0}\rangle \nonumber \\
        =&\langle \Phi _{0}|e ^{\hat{\kappa}}[\mathcal{H},E _{jb}^{-}]e ^{-\hat{\kappa}}|\Phi _{0}\rangle \nonumber \\
        =&\langle \Phi _{0}(\hat{\kappa})|[\mathcal{H},E _{jb}^{-}]|\Phi _{0}(\hat{\kappa})\rangle
    \end{align}
    The second term:\\
    \textcolor{magenta}{(TODO: show that $\hat{\Lambda}$ and $\hat{\kappa}$ commute.)}

    \subsection{Orthonormality Condition}





\end{document}