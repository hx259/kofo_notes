\documentclass[a4paper,11pt]{article}
\usepackage{braket}
\usepackage{physics}
\usepackage{parskip}
\usepackage{bm}
\usepackage{float}
\usepackage[utf8]{inputenc}
\usepackage{amsmath}
\usepackage{pgfplots}
\usepackage{mathrsfs}
\usepackage{simpler-wick}
\usepackage{enumerate}
\usepackage{csquotes}
\usepackage{subcaption}
\usepackage{fancyhdr}
\usepackage{tablefootnote}
\usepackage{microtype}
\usepackage{cleveref}
\usepackage{titling}
\usepackage{erewhon}
\usepackage[a4paper,width=150mm,top=25mm,bottom=25mm]{geometry}
\usepackage[Sonny]{fncychap}
\usepackage[calcwidth]{titlesec}


\usetikzlibrary{shapes.geometric, arrows}

\allowdisplaybreaks

\pgfplotsset{compat=1.7}

%fncychap layout (for chapter page)
%\renewcommand{\thechapter}{\Roman{chapter}}
\ChNameVar{\bfseries\LARGE\fontfamily{phv}}
\ChNumVar{\fontsize{50}{52}\usefont{OT1}{ptm}{m}{n}\selectfont}
\ChTitleVar{\LARGE\rm\bfseries}
\ChRuleWidth{0.8pt}

%fancyhdr layout (for header and footer)
\pagestyle{fancy}
\fancyhead{}
\fancyhead[C]{\fontsize{9}{9}\itshape{\rightmark}}
\setlength{\headheight}{15pt}

\begin{document}

\section{Contravariant Space}
By 2nd order perturbation theory, the most important contribution to the corrlation energy
comes from the configuration $\Phi _{I}$ which spans the first-order interacting space:
\begin{equation}
    \langle \Phi _{I}|\hat{H}|0\rangle \neq 0
\end{equation}
In RHF, $|0\rangle=|\Phi _{\text{HF}}\rangle$ is the HF determinant, and the singlet excitation operators are defined as:
\begin{align}
    \hat{E} _{i}^{a} =& \sum_{\sigma}\hat{a}^{\dagger}_{\sigma}\hat{i}_{\sigma} \\
    \hat{e} _{ij}^{ab} =& \sum_{\sigma \tau}\hat{a}^{\dagger}_{\sigma}\hat{b}^{\dagger}_{\tau}\hat{j}_{\tau}\hat{i}_{\sigma} = \hat{E}_{i}^{a}\hat{E}_{j}^{b}
\end{align}
hence the singly and doubly excitation configurations are:
\begin{align}
    |\Phi _{i}^{a}\rangle =& \hat{E}_{i}^{a}|0\rangle \\
    |\Phi _{ij}^{ab}\rangle =& \hat{E}_{i}^{a}\hat{E}_{j}^{b}|0\rangle
\end{align}
From Brillouin theorem, i.e. $f _{ai}=0$, we know that the first-order wavefunction is a linear combination of the doubly excitation configurations:
\begin{equation}
    |\Psi ^{(1)}\rangle = \frac{1}{2}\sum_{ij,ab} T _{ij}^{ab}|\Phi _{ij}^{ab}\rangle
\end{equation}
the $\frac{1}{2}$ arises due to the summation in spin \textcolor{red}{(?)}.
\\

It is convenient to define a set of contravariant configurations:
\begin{align}
    |\tilde{\Phi}_{i}^{a}\rangle =& \frac{1}{2}\hat{E} _{i}^{a}|0\rangle =\frac{1}{2}|\Phi _{i}^{a}\rangle \\
    |\tilde{\Phi}_{ij}^{ab}\rangle =& \frac{1}{6}(2 \hat{E} _{i}^{a}\hat{E} _{j}^{b} + \hat{E} _{j}^{a}\hat{E} _{i}^{b})|0\rangle =\frac{1}{6}(2 |\Phi _{ij}^{ab}\rangle + |\Phi _{ji}^{ab}\rangle)
\end{align}
As a result, some expressions become simpler:
\begin{align}
    \langle \tilde{\Phi}_{ij}^{ab}|\Phi _{kl}^{cd}\rangle =& \delta _{ac}\delta _{bd}\delta _{ik}\delta _{jl} + \delta _{ad}\delta _{bc}\delta _{il}\delta _{jk} \\
    \langle \tilde{\Phi}_{ij}^{ab}|\Psi ^{(1)}\rangle =& T _{ij}^{ab} \\
    \langle \tilde{\Phi}_{ij}^{ab}|\hat{H}|\Psi ^{(0)}\rangle =& \langle ab|ij\rangle
\end{align}
Now $|\Psi ^{(1)}\rangle$ becomes:
\begin{equation}
    |\Psi ^{(1)}\rangle = \frac{1}{2}\sum_{ij,ab}T _{ij}^{ab}|\Phi _{ij}^{ab}\rangle = \sum_{ij,ab}\tilde{T}_{ij}^{ab}|\tilde{\Phi}_{ij}^{ab}\rangle
\end{equation}
where $\tilde{T} _{ij}^{ab} = 2T _{ij}^{ab} - T _{ji}^{ab}$.



\section*{*Derivation}
Derivation for eqns. (9), (10) and (11):
\begin{equation}
    \langle \tilde{\Phi}_{ij}^{ab}|\Phi _{kl}^{cd}\rangle = \langle 0|\frac{1}{6}(2 \hat{E}_{i}^{a}\hat{E} _{j}^{b} + \hat{E} _{j}^{a}\hat{E} _{i}^{b})\hat{E} _{k}^{c} \hat{E}_{l}^{d}|0 \rangle
\end{equation}
in which:
\begin{align}
    \hat{E}_{i}^{a}\hat{E}_{j}^{b} =& (\hat{a}^{\dagger}_{\alpha}\hat{i}_{\alpha} + \hat{a}^{\dagger}_{\beta}\hat{i}_{\beta})(\hat{b}^{\dagger}_{\alpha}\hat{j}_{\alpha}+ \hat{b}^{\dagger}_{\beta}\hat{j}_{\beta}) \nonumber \\
    =& \hat{a}^{\dagger}_{\alpha}(\hat{i}_{\alpha}\hat{b}^{\dagger}_{\alpha})\hat{j}_{\alpha} + \hat{a}^{\dagger}_{\alpha}(\hat{i}_{\alpha}\hat{b}^{\dagger}_{\beta})\hat{j}_{\beta} 
    + \hat{a}^{\dagger}_{\beta}(\hat{i}_{\beta}\hat{b}^{\dagger}_{\alpha})\hat{j}_{\alpha} + \hat{a}^{\dagger}_{\beta}(\hat{i}_{\beta}\hat{b}^{\dagger}_{\beta})\hat{j}_{\beta}
\end{align}
by $[\hat{p}^{\dagger},\hat{q}]_{+} = \delta _{pq}$, we have:
\begin{equation}
    \hat{E}_{i}^{a}\hat{E}_{j}^{b} = -\hat{a}^{\dagger}_{\alpha}\hat{b}^{\dagger}_{\alpha}\hat{i}_{\alpha}\hat{j}_{\alpha} - \hat{a}^{\dagger}_{\alpha}\hat{b}^{\dagger}_{\beta}\hat{i}_{\alpha}\hat{j}_{\beta}
    - \hat{a}^{\dagger}_{\beta}\hat{b}^{\dagger}_{\alpha}\hat{i}_{\beta}\hat{j}_{\alpha} - \hat{a}^{\dagger}_{\beta}\hat{b}^{\dagger}_{\beta}\hat{i}_{\beta}\hat{j}_{\beta}
\end{equation}
Similarly:
\begin{equation}
    \hat{E}_{k}^{c}\hat{E}_{l}^{d} =-\hat{c}^{\dagger}_{\alpha}\hat{d}^{\dagger}_{\alpha}\hat{k}_{\alpha}\hat{l}_{\alpha} - \hat{c}^{\dagger}_{\alpha}\hat{d}^{\dagger}_{\beta}\hat{k}_{\alpha}\hat{l}_{\beta}
    - \hat{c}^{\dagger}_{\beta}\hat{d}^{\dagger}_{\alpha}\hat{k}_{\beta}\hat{l}_{\alpha} - \hat{c}^{\dagger}_{\beta}\hat{d}^{\dagger}_{\beta}\hat{k}_{\beta}\hat{l}_{\beta}
\end{equation}



\end{document}