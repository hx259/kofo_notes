\documentclass[a4paper,11pt]{article}
\usepackage{braket}
\usepackage{physics}
\usepackage{parskip}
\usepackage{bm}
\usepackage{float}
\usepackage[utf8]{inputenc}
\usepackage{amsmath}
\usepackage{pgfplots}
\usepackage{mathrsfs}
\usepackage{simpler-wick}
\usepackage{enumerate}
\usepackage{csquotes}
\usepackage{subcaption}
\usepackage{fancyhdr}
\usepackage{tablefootnote}
\usepackage{microtype}
\usepackage{cleveref}
\usepackage{titling}
\usepackage{erewhon}
\usepackage[a4paper,width=150mm,top=25mm,bottom=25mm]{geometry}
\usepackage[Sonny]{fncychap}
\usepackage[calcwidth]{titlesec}

\usetikzlibrary{shapes.geometric, arrows}

\allowdisplaybreaks

\pgfplotsset{compat=1.7}

%fncychap layout (for chapter page)
%\renewcommand{\thechapter}{\Roman{chapter}}
\ChNameVar{\bfseries\LARGE\fontfamily{phv}}
\ChNumVar{\fontsize{50}{52}\usefont{OT1}{ptm}{m}{n}\selectfont}
\ChTitleVar{\LARGE\rm\bfseries}
\ChRuleWidth{0.8pt}

%fancyhdr layout (for header and footer)
\pagestyle{fancy}
\fancyhead{}
\fancyhead[C]{\fontsize{9}{9}\itshape{\rightmark}}
\setlength{\headheight}{15pt}

\begin{document}

\section{UHF}


\section{RHF}

\subsection{Singlet Excitation Operators}
    Generally, the one- and two-body excitation operators are defined as:
    \begin{align}
        E _{q}^{p} &= \hat{p}^{\dagger}_{\alpha}\hat{q}_{\alpha} + \hat{p}^{\dagger}_{\beta}\hat{q}_{\beta}  \\
        e _{rs}^{pq} &= \sum_{\sigma \tau}\hat{p}^{\dagger}_{\sigma}\hat{q}^{\dagger}_{\tau}\hat{s}_{\tau}\hat{r}_{\sigma}
    \end{align}

    By:
    \begin{equation}
        [\hat{p}^{\dagger},\hat{q}]_{+} = \hat{\delta}_{pq}
    \end{equation}
    the two-body singlet excitation operator could be written as:
    \begin{align}
        e ^{pq}_{rs} =& \sum_{\sigma \tau}\hat{p}^{\dagger}_{\sigma}\hat{q}^{\dagger}_{\tau}\hat{s}_{\tau}\hat{r}_{\sigma} \nonumber \\
        =& -\sum_{\sigma \tau}\hat{p}^{\dagger}_{\sigma}\hat{q}^{\dagger}_{\tau}\hat{r}_{\sigma}\hat{s}_{\tau} \nonumber \\
        =& -\sum_{\sigma \tau}\hat{p}^{\dagger}_{\sigma}(\delta _{qr}\delta _{\sigma \tau} - \hat{r}_{\sigma}\hat{q}^{\dagger}_{\tau})\hat{s}_{\tau} \nonumber \\
        =& \sum_{\sigma \tau}\hat{p}^{\dagger}_{\sigma}\hat{r}_{\sigma}\hat{q}^{\dagger}_{\tau}\hat{s}_{\tau} - \sum_{\sigma \tau}\delta _{qr}\delta _{\sigma \tau}\hat{p}^{\dagger}_{\sigma}\hat{s}_{\tau} \nonumber \\
        =& \hat{p}^{\dagger}_{\alpha}\hat{r}_{\alpha}\hat{q}^{\dagger}_{\alpha}\hat{s}_{\alpha} + \hat{p}^{\dagger}_{\alpha}\hat{r}_{\alpha}\hat{q}^{\dagger}_{\beta}\hat{s}_{\beta}
        + \hat{p}^{\dagger}_{\beta}\hat{r}_{\beta}\hat{q}^{\dagger}_{\alpha}\hat{s}_{\alpha} + \hat{p}^{\dagger}_{\beta}\hat{r}_{\beta}\hat{q}^{\dagger}_{\beta}\hat{s}_{\beta}
        -\delta _{qr}(\hat{p}^{\dagger}_{\alpha}\hat{s}_{\alpha} + \hat{p}^{\dagger}_{\beta}\hat{s}_{\beta}) \nonumber \\
        =& E _{r}^{p}E ^{q}_{s} - \delta _{qr}E _{s}^{p}
    \end{align} 
    Note, conventionally, the indices for the two-body operator are written in Chemists' notation.
    \\
    
    Considering the nature of excitation operators, the following expressions are often more useful:

    \begin{align}
        E _{i}^{a} &= \hat{a}^{\dagger}_{\alpha}\hat{i}_{\alpha} + \hat{a}^{\dagger}_{\beta}\hat{i}_{\beta} \\
        e _{ij}^{ab} &= \sum_{\sigma \tau}\hat{a}^{\dagger}_{\sigma}\hat{b}^{\dagger}_{\tau}\hat{j}_{\tau}\hat{i}_{\sigma}
        = E _{i}^{a}E _{j}^{b}
    \end{align}





\end{document}