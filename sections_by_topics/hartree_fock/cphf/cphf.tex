\documentclass[a4paper,11pt]{article}
\usepackage{braket}
\usepackage{physics}
\usepackage{parskip}
\usepackage{bm}
\usepackage{float}
\usepackage[utf8]{inputenc}
\usepackage{amsmath}
\usepackage{pgfplots}
\usepackage{mathrsfs}
\usepackage{simpler-wick}
\usepackage{enumerate}
\usepackage{csquotes}
\usepackage{subcaption}
\usepackage{fancyhdr}
\usepackage{tablefootnote}
\usepackage{microtype}
\usepackage{cleveref}
\usepackage{titling}
\usepackage{erewhon}
\usepackage[a4paper,width=150mm,top=25mm,bottom=25mm]{geometry}
\usepackage[Sonny]{fncychap}
\usepackage[calcwidth]{titlesec}

\usetikzlibrary{shapes.geometric, arrows}

\allowdisplaybreaks

\pgfplotsset{compat=1.7}

%fncychap layout (for chapter page)
%\renewcommand{\thechapter}{\Roman{chapter}}
\ChNameVar{\bfseries\LARGE\fontfamily{phv}}
\ChNumVar{\fontsize{50}{52}\usefont{OT1}{ptm}{m}{n}\selectfont}
\ChTitleVar{\LARGE\rm\bfseries}
\ChRuleWidth{0.8pt}

%fancyhdr layout (for header and footer)
\pagestyle{fancy}
\fancyhead{}
\fancyhead[C]{\fontsize{9}{9}\itshape{\rightmark}}
\setlength{\headheight}{15pt}

\begin{document}
\section{Coupled Perturbed Hatree-Fock (CPHF) Method}
    This is a summary of the article Pople et al., IJQC (1979).
    \\

    The fock equation, subject to a perturbation $\xi$, in matrix notation:
    \begin{equation}
        \mathbf{F}(\xi)\mathbf{C}(\xi) = \mathbf{S}(\xi))\mathbf{C}(\xi)\mathbf{E}(\xi)
    \end{equation}
    The spin-orbitals are required to be orthonormal for all values of $\xi$:
    \begin{equation}
        \mathbf{C}^{\dagger}(\xi)\mathbf{S}(\xi)\mathbf{C}(\xi) = \mathbf{I}
    \end{equation}
    We would like to transform $\mathbf{C}(\xi)$ to unperturbed spin-orbital coefficients, i.e. $\mathbf{C}(0)$.
    Consider the transformation from AO to MO:
    \begin{equation}
        \psi _{p}(\xi) = \sum_{\mu}C _{\mu p}(\xi)\phi _{\mu}(\xi)
    \end{equation}
    Suppose we have the basis functions $\{\phi _{i}(0)\}$ changed to $\{\phi _{i}(\xi)\}$ but the coefficients $\mathbf{C}(0)$ remains unperturbed:
    \begin{equation}
        \psi _{p}(\xi) = \sum_{q}U _{qp}(\xi)\big(\sum_{\mu}C _{\mu q}(0)\phi _{\mu}(\xi)\big)
    \end{equation}
    The matrix $\mathbf{U}(\xi)$ transforms the spin-orbital coefficients $\mathbf{C}(0)$  to a perturbed basis:
    \begin{align}
        C _{\mu p}(\xi) =& \sum_{q}C _{\mu q}(0)U _{qp}(\xi) \\
        \Updownarrow & \nonumber \\
        \mathbf{C}(\xi) =& \mathbf{C}(0)\mathbf{U}(\xi)
    \end{align}
    Now our aim is to find $\mathbf{U}(\xi)$ for perturbation $\xi$.\\

    The Roothaan equations subject to perturbation are:
    \begin{equation}
        \mathbf{F}(\xi)\mathbf{C}(\xi) = \mathbf{S}(\xi)\mathbf{C}(\xi)\mathbf{E}(\xi)
    \end{equation}
    Writing the equation in terms of unperturbed coefficients and left-multiply by $\mathbf{C}^{\dagger}(0)$:
    \begin{align}
        \mathbf{F}(\xi)\mathbf{C}(0)\mathbf{U}(\xi) =& \mathbf{S}(\xi)\mathbf{C}(0)\mathbf{U}(\xi)\mathbf{E}(\xi) \nonumber \\
        \Leftrightarrow \mathbf{C}^{\dagger}(0)\mathbf{F}(\xi)\mathbf{C}(0)\mathbf{U}(\xi) =& \mathbf{C}(0)\mathbf{S}(\xi)\mathbf{C}(0)\mathbf{U}(\xi)\mathbf{E}(\xi)
    \end{align} 
    By defining:
    \begin{align}
        \bm{\mathcal{F}}(\xi) &= \mathbf{C}^{\dagger}(0)\mathbf{F}(\xi)\mathbf{C}(0)  \\
        \bm{\mathcal{S}}(\xi) &= \mathbf{C}^{\dagger}(0)\mathbf{S}(\xi)\mathbf{C}(0) 
    \end{align}
    the Roothaan equations become:
    \begin{equation}
        \bm{\mathcal{F}}(\xi)\mathbf{U}(\xi) = \bm{\mathcal{S}}(\xi)\mathbf{U}(\xi)\mathbf{E}(\xi)
    \end{equation}
    and the orthonomality condition follows:
    \begin{equation}
        \mathbf{U}^{\dagger}(\xi)\bm{\mathcal{S}}(\xi)\mathbf{U}(\xi) = \mathbf{I}
    \end{equation}
    Note here that, by definition, $\bm{\mathcal{S}}(0)$ and $\mathbf{U}(0)$ are the unit matrix $\mathbf{I}$:
    \begin{align}
        \mathbf{C}(\xi = 0) = \mathbf{C}(0)\mathbf{U}(\xi = 0) \Longleftrightarrow \mathbf{U}(0) = \mathbf{I}  \\
        \bm{\mathcal{S}}(\xi = 0) = \mathbf{C}^{\dagger}(0)\mathbf{S}(\xi = 0)\mathbf{C}(0) = \mathbf{I}
    \end{align}
    Turn off the perturbation in the Roothaan equations subject to perturbation:
    \begin{align}
         \bm{\mathcal{F}}(0)\mathbf{U}(0) =& \bm{\mathcal{S}}(0)\mathbf{U}(0)\mathbf{E}(0) \\
         \Updownarrow& \nonumber \\
         \bm{\mathcal{F}}(0) =& \mathbf{E}(0)
    \end{align}
    With the unperturbed case defined, we now try to solve the Roothaan equations subject to perturbation, with orthonomality condition, by expanding the perturbed matrices in power series:
    \begin{align}
        \bm{\mathcal{F}}(\xi) =& \mathbf{E}(0) + \xi \bm{\mathcal{F}}^{(1)} + \order{\xi ^{2}} \\
        \bm{\mathcal{S}}(\xi) =& \mathbf{I} + \xi \bm{\mathcal{S}}^{(1)} + \order{\xi ^{2}} \\
        \mathbf{U}(\xi) =& \mathbf{I} + \xi \mathbf{U}^{(1)} + \order{\xi ^{2}} \\
        \mathbf{E}(\xi) =& \mathbf{E}(0) + \xi \mathbf{E}^{(1)} + \order{\xi ^{2}}
    \end{align}
    Substitute these expansions into the Roothaan equations and normalisation condition, and collect linear terms in $\xi$:
    \begin{align}
        \mathbf{E}(0)\mathbf{U}^{(1)} + \bm{\mathcal{F}}^{(1)} =& \mathbf{E}^{(1)} + \mathbf{U}^{(1)}\mathbf{E}(0) + \bm{\mathcal{S}}^{(1)}\mathbf{E}(0) \\
        \mathbf{0} =& \mathbf{U}^{(1)\dagger} + \bm{\mathcal{S}}^{(1)} + \mathbf{U}^{(1)}
    \end{align}
    Now our aim is to solve for $\mathbf{U}^{(1)}$ and $\mathbf{E}^{(1)}$.\\
    As adding a random phase factor to the wavefunction does not alter the physical observables (e.g. electron density), we can safely choose elements of $\mathbf{U}$ to be real, WLOG.
    Therefore, considering the diagonal terms in the orthonomality equation:
    \begin{equation}
        U _{pp} = -\frac{1}{2}\mathcal{S}_{pp}
    \end{equation}
    By looking at the diagonal terms of the Roothaan equations:
    \begin{align}
        \big(\mathcal{F}^{(1)} + E (0)U ^{(1)}\big)_{pp} =& \big(\mathcal{S}^{(1)}E(0) + U ^{(1)}E(0) + E ^{(1)} \big)_{pp}  \nonumber \\
        \Leftrightarrow \mathcal{F}^{(1)}_{pp} + E_{p q}(0) U ^{(1)}_{q p} =& \mathcal{S}^{(1)}_{p r}E ^{(0)}_{r p} + U ^{(1)}_{p s}E _{s p}(0) + E _{pp}^{(1)} \nonumber \\
        \Leftrightarrow \mathcal{F}^{(1)}_{pp} + E_{p}(0)\delta _{p q} U ^{(1)}_{q p} =& \mathcal{S}^{(1)}_{p r}E _{p}(0)\delta _{r p} + U ^{(1)}_{p s}E _{p}(0)\delta _{p s} + E _{pp}^{(1)} \nonumber \\
        \Leftrightarrow \mathcal{F}^{(1)}_{pp} + E_{p}(0)U _{pp}^{(1)} =& \mathcal{S}_{pp}^{(1)}E _{p}(0) + U _{pp}^{(1)}E _{p}(0) + E _{p}^{(1)} \nonumber \\
        \Downarrow& \nonumber \\
        E _{p}^{(1)} =& \mathcal{F}_{pp}^{(1)} - \mathcal{S}_{pp}^{(1)}E _{p}(0)
    \end{align}        
    Einstein's summation convention are assumed here as well as in later sections, unless otherwise noted.
    \\
    To find the off-diagonal element of $\mathbf{U}$, we look at the off-diagonal terms in the Roothaan equations:
    \begin{align}
        \big(\mathcal{F}^{(1)} + E (0)U ^{(1)}\big)_{pq} =& \big(\mathcal{S}^{(1)}E(0) + U ^{(1)}E(0) + E ^{(1)} \big)_{pq}  \nonumber \\
        \Leftrightarrow \mathcal{F}_{pq}^{(1)} + E _{pr}(0)U _{rq}^{(1)} =& \mathcal{S}_{ps}^{(1)}E _{sq}(0) + U _{pt}^{(1)}E _{tq}(0) + E _{pq}^{(1)} \nonumber \\
        \Leftrightarrow \mathcal{F}_{pq}^{(1)} + E _{p}(0)\delta _{pr} U _{rq}^{(1)} =& \mathcal{S}_{ps}^{(1)}E _{q}(0)\delta _{sq} + U _{pt}^{(1)}E _{q}(0)\delta _{tq} + E _{pq}^{(1)} \nonumber \\
        \Downarrow& \nonumber \\
        U _{pq}^{(1)} =& \frac{\mathcal{F}_{pq}^{(1)} - \mathcal{S}_{pq}^{(1)}E _{q}(0)}{E _{q}(0) - E _{p}(0)}
    \end{align}
    \textcolor{cyan}{TODO:This is unfinished.}


\end{document}