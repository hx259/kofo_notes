\documentclass[a4paper,11pt]{article}
\usepackage{braket}
\usepackage{physics}
\usepackage{parskip}
\usepackage{bm}
\usepackage{float}
\usepackage[utf8]{inputenc}
\usepackage{amsmath}
\usepackage{pgfplots}
\usepackage{mathrsfs}
\usepackage{simpler-wick}
\usepackage{enumerate}
\usepackage{csquotes}
\usepackage{subcaption}
\usepackage{fancyhdr}
\usepackage{tablefootnote}
\usepackage{microtype}
\usepackage{cleveref}
\usepackage{titling}
\usepackage{erewhon}
\usepackage[a4paper,width=150mm,top=25mm,bottom=25mm]{geometry}
\usepackage[Sonny]{fncychap}
\usepackage[calcwidth]{titlesec}

\usetikzlibrary{shapes.geometric, arrows}

\allowdisplaybreaks

\pgfplotsset{compat=1.7}

%fncychap layout (for chapter page)
%\renewcommand{\thechapter}{\Roman{chapter}}
\ChNameVar{\bfseries\LARGE\fontfamily{phv}}
\ChNumVar{\fontsize{50}{52}\usefont{OT1}{ptm}{m}{n}\selectfont}
\ChTitleVar{\LARGE\rm\bfseries}
\ChRuleWidth{0.8pt}

%fancyhdr layout (for header and footer)
\pagestyle{fancy}
\fancyhead{}
\fancyhead[C]{\fontsize{9}{9}\itshape{\rightmark}}
\setlength{\headheight}{15pt}

\begin{document}

\section{Paramagnetic and Diamagnetic Interactions}
    \begin{itemize}
    \item Paramagnetic interactions are (linear) $1 ^{\text{st}}$ order: $\sum_{i}\mathbf{A}_{i}(\mathbf{r}_{i})\cdot \mathbf{p}_{i} + \sum_{i}\mathbf{B}(\mathbf{r}_{i})\cdot \mathbf{s}_{i}$ 
    \item Energies of paramagnetic systems could be lowered by aligning the molecular magnetic moment with the field
    \item Diamagnetic interactions are $2 ^{\text{nd}}$ order: $+\frac{1}{2}\sum_{i}A ^{2}(\mathbf{r}_{i})$ 
    \item Energy of diamagnetic systems could only be increased with the field
    \end{itemize}
    Vector potential has both contribution from external magnetic field $\mathbf{B}$ and nuclear magnetic field $\mathbf{B}_{K}$:
    \begin{equation}
        \mathbf{A}_{i} = \mathbf{A}_{0}(\mathbf{r}_{i}) + \sum_{K}\mathbf{A}_{K}(\mathbf{r}_{i})
    \end{equation} 
    with:
    \begin{align}
        \mathbf{A}_{0}(\mathbf{r}_{i}) &= \frac{1}{2}\mathbf{B}\times \mathbf{r}_{0} &\mathbf{B} &= \mathbf{\nabla}\times \mathbf{A}(\mathbf{r}) \\
        \mathbf{A}_{K}(\mathbf{r}_{i}) &= \alpha ^{2}\frac{\mathbf{M}_{K}\times \mathbf{r}_{K}}{r _{K}^{3}} &\mathbf{B}_{K}(\mathbf{r}) &= \mathbf{\nabla}\times \mathbf{A}_{K}(\mathbf{r})
    \end{align}

    \subsection{Zeeman Paramagnetic Interactions}
    \begin{align*}
        \text{Zeeman} &= 1 ^{\text{st}} \text{order interaction of electron with external magnetic field } \mathbf{B} \\
               &= \text{Orbital Zeeman} + \text{Spin Zeeman}
    \end{align*}
    \begin{equation}
        H _{\text{zm}}^{(1)} = \sum_{i}(\mathbf{A}_{0}(\mathbf{r}_{i})\cdot \mathbf{p}_{i} + \mathbf{B}\cdot \mathbf{s}_{i})
    \end{equation}
    Orbital Zeeman:
    \begin{align}
        \sum_{i}\mathbf{A}_{0}(\mathbf{r _{i}})\cdot \mathbf{p}_{i} &= \frac{1}{2}\sum_{i}\mathbf{B}\times \mathbf{r}_{i0}\cdot \mathbf{p}_{i} \nonumber \\
        &= \frac{1}{2}\sum_{i}\mathbf{B}\cdot \mathbf{l}_{i0} \\ \nonumber
        &= \frac{1}{2}\mathbf{B}\cdot \mathbf{L}_{0}
    \end{align}
    in which $\mathbf{l}_{i0} = \mathbf{r}_{i0}\times \mathbf{p}_{i}$ is the orbital angular momentum and $\mathbf{L}_{0}=\sum_{i}\mathbf{l}_{i0}$ is the total angular momentum, relative to $\mathbf{R}_{0}$.\\
    We also used the permutation relation of triple product: $\mathbf{a} \cdot (\mathbf{b}\times \mathbf{c}) = \mathbf{b} \cdot (\mathbf{c}\times \mathbf{a})$
    \break
    \\
    Spin Zeeman:
    \begin{equation}
        \sum_{i}\mathbf{B}\cdot \mathbf{s}_{i} = \mathbf{B}\cdot \mathbf{S}
    \end{equation}
    where $\mathbf{S} = \sum_{i}\mathbf{s}_{i}$ is the total spin angular momentum. \\

    Therefore, the total Zeeman Hamiltonian is:
    \begin{equation}
        H_{\text{zm}}^{(1)} = \mathbf{B}\cdot \mathbf{H}_{0}^{\text{zm}} = \mathbf{B}\cdot (\frac{1}{2}\mathbf{L}_{0} + \mathbf{S})
    \end{equation}

    \subsection{Hyperfine Paramagnetic Interactions}
    \begin{align*}
        \text{Hyperfine} &= 1 ^{\text{st}} \text{order interaction of electron with nuclear magnetic field, with magnetic moment } \mathbf{M}_{K} \\
        &= \text{Orbital Hyperfine} + \text{Spin Hyperfine}
    \end{align*}
    \begin{equation}
        H _{\text{hf}}^{(1)} = \sum_{i,K}(\mathbf{A}_{K}(\mathbf{r}_{i})\cdot \mathbf{p}_{i} + \mathbf{B}_{K}(\mathbf{r}_{i})\cdot \mathbf{s}_{i})
    \end{equation}
    where $\mathbf{A}_{K}(\mathbf{r})=\alpha ^{2}\frac{\mathbf{M}_{K}\times \mathbf{r}_{K}}{r _{K}^{3}}$ is the vector field associated with nuclear magnetic field $\mathbf{B}_{K}$. \\

    Orbital Hyperfine:
    \begin{equation}
        \sum_{i,K} \mathbf{A}_{K}(\mathbf{r}_{i})\cdot \mathbf{p}_{i} = \alpha ^{2} \sum_{i,K}\frac{(\mathbf{M}_{K}\times \mathbf{r}_{iK})\cdot \mathbf{p}_{i}}{r _{iK}^{3}} = \alpha ^{2}\sum_{i,K}\frac{\mathbf{M}_{K}\cdot \mathbf{l}_{iK}}{r _{iK}^{3}}
    \end{equation}
    Spin Hyperfine:
    \begin{equation}
        \sum_{i,K} \mathbf{B}_{K}(\mathbf{r}_{i})\cdot \mathbf{s}_{i}
    \end{equation}
    and the nuclear magnetic field is:
    \begin{equation}
        \mathbf{B}_{K}(\mathbf{r}) = \mathbf{\nabla}\times \mathbf{A}_{K}(\mathbf{r}) = \frac{8 \pi \alpha ^{2}}{3}\delta (\mathbf{r}_{K})\mathbf{M}_{K} + \alpha ^{2}\frac{3 \mathbf{r}_{K}(\mathbf{r}_{K}\cdot \mathbf{M}_{K})- r _{K}^{2}\mathbf{M}_{K}}{r _{K}^{5}}
    \end{equation} 
    in which the first term is contact term and only contributes when electron is at the nucleus, while the second term, the dipole term, contributes at distance. \\

    Therefore, the total Hyperfine Hamiltonian is:
    \begin{align}
        H _{\text{hf}}^{(1)} &= \sum_{i,K}(\mathbf{A}_{K}(\mathbf{r}_{i})\cdot \mathbf{p}_{i} + \mathbf{B}_{K}(\mathbf{r}_{i})\cdot \mathbf{s}_{i})  \nonumber \\
        &= \alpha ^{2} \sum_{i,K}\frac{\mathbf{M}_{K}\cdot \mathbf{l}_{iK}}{r _{iK}^{3}} + \frac{8 \pi \alpha ^{2}}{3}\sum_{i,K}\delta _{\mathbf{r}_{iK}}\mathbf{M}_{K}\cdot \mathbf{s}_{i} + \alpha ^{2} \sum_{i,K}\mathbf{M}_{K}\cdot \frac{3 \mathbf{r}_{iK}\mathbf{r}_{iK}^{\text{T}}- \mathbf{r}_{iK}^{2}\mathbf{I}_{3}}{r _{iK}^{5}} \mathbf{s}_{i} \\ \nonumber
        &= \sum_{K}\mathbf{M}_{K}\cdot (\alpha ^{2}\sum_{i}\frac{\mathbf{l}_{iK}}{r _{iK}^{3}} + \frac{8 \pi \alpha ^{2}}{3}\sum_{i}\delta (\mathbf{r}_{iK})\mathbf{s}_{i} + \alpha ^{2}\sum_{i}\frac{3 \mathbf{r}_{iK}\mathbf{r}_{iK}^{\text{T}}-r _{iK}^{2}\mathbf{I}_{3}}{r _{iK}^{5}} \mathbf{s}_{i}) \\ \nonumber
        &= \sum_{K}\mathbf{M}_{K}\cdot (\mathbf{H}_{K}^{\text{PSO}} + \mathbf{H}_{K}^{\text{FC}} + \mathbf{H}_{K}^{\text{SD}}) \\ \nonumber
        &= \sum_{K}\mathbf{M}_{K}\cdot \mathbf{H}_{K}^{\text{hf}}
    \end{align}
    in which:
    \begin{align}
        \mathbf{H}_{K}^{\text{PSO}} &= \alpha ^{2} \sum_{i}\frac{\mathbf{l}_{iK}}{r _{iK}^{3}} = \text{paramagnetic spin-orbit operator} \\
        \mathbf{H}_{K}^{\text{FC}} &= \frac{8 \pi \alpha ^{2}}{3} \sum_{i}\delta (\mathbf{r}_{iK})\mathbf{s}_{i} = \text{Fermi-contact operator} \\
        \mathbf{H}_{K}^{\text{SD}} &= \alpha ^{2}\sum_{i}\frac{3 \mathbf{r}_{iK}\mathbf{r}_{iK}^{\text{T}}-r _{iK}^{2}\mathbf{I}_{3}}{r _{iK}^{5}}\mathbf{s}_{i} = \text{spin-dipole operator}
    \end{align}
    where $\mathbf{H}_{K}^{\text{PSO}}$ comes from the Orbital-Hyperfine contribution and $\mathbf{H}_{K}^{\text{FC}}$,$\mathbf{H}_{K}^{\text{SD}}$ come from the Spin-Hyperfine contribution.   

    \subsection{Diamagnetic Interactions}
    Recall:
    \begin{equation}
        H = H ^{(0)} + H ^{(1)} + H ^{(2)}
    \end{equation}
    where $H ^{(2)} = \frac{1}{2}\sum_{i}\mathbf{A}_{i}^{2}(\mathbf{r}_{i})$ is the diamagnetic contribution to the Hamiltonian.\\ 
    The vector potential could be splitted into contributions from the external magnetic field $\mathbf{B}$ and nuclear magnetic field $\mathbf{B}_{K}$: 
    \begin{equation}
        \mathbf{A}_{i} = \mathbf{A}_{0}(\mathbf{r}_{i}) + \sum_{K}\mathbf{A}_{K}(\mathbf{r}_{i})
    \end{equation}
    To evaluate the diamagnetic interaction:
    \begin{align}
        H _{\text{dia}^{(2)}} &= \frac{1}{2}\sum_{i}A _{i}^{2}(\mathbf{r}_{i})  \nonumber \\
         &= \frac{1}{2}\sum_{i}\big(\mathbf{A}_{0}(\mathbf{r}_{i}) + \sum_{K}\mathbf{A}_{K}(\mathbf{r}_{i})\big)\cdot\big(\mathbf{A}_{0}(\mathbf{r}_{i}) + \sum_{K}\mathbf{A}_{K}(\mathbf{r}_{i})\big) \\ \nonumber
         &= \frac{1}{2}\sum_{i}A _{0}^{2}(\mathbf{r}_{i}) + \sum_{i,K}\mathbf{A}_{K}(\mathbf{r}_{i})\cdot \mathbf{A}_{0}(\mathbf{r}_{i}) + \frac{1}{2}\sum_{i,K,L}\mathbf{A}_{K}(\mathbf{r}_{i})\cdot \mathbf{A}_{L}(\mathbf{r}_{i}) \\
    \end{align}
    Evaluate term by term:
    \begin{align}
        \frac{1}{2}\sum_{i}A _{0}^{2}(\mathbf{r}_{i}) &= \frac{1}{2} \sum_{i}\frac{1}{2}(\mathbf{B}\times \mathbf{r}_{i0})\cdot \frac{1}{2}(\mathbf{B}\times \mathbf{r}_{i0}) \nonumber \\
         &= \mathbf{B}^{\text{T}}(\frac{1}{8}\sum_{i}\mathbf{I}_{3}r _{i0}^{2}- r _{i0}r _{i0}^{\text{T}})\mathbf{B} \\ \nonumber
         &= \mathbf{B}^{\text{T}}\mathbf{H}_{00}^{\text{dia}}\mathbf{B} 
    \end{align}
    \begin{align}
        \sum_{i,K}\mathbf{A}_{K}(\mathbf{r}_{i})\cdot \mathbf{A}_{0}(\mathbf{r}_{i}) &= \frac{1}{2} \alpha ^{2}\sum_{i,K}\frac{(\mathbf{B}\times \mathbf{r}_{i0})\cdot (\mathbf{M}_{K}\times \mathbf{r}_{iK})}{r _{iK}^{3}}  \nonumber \\
         &= \mathbf{M}_{K}^{\text{T}} \sum_{K}(\frac{\alpha ^{2}}{2}\sum_{i}\frac{\mathbf{r}_{i0}^{\text{T}}\mathbf{r}_{iK}\mathbf{I}_{3}-\mathbf{r}_{i0}\mathbf{r}_{iK}^{\text{T}}}{r _{iK}^{3}})\mathbf{B} \nonumber \\
         &= \sum_{K}\mathbf{M}_{K}^{\text{T}}\mathbf{H}_{K0}^{\text{dia}}\mathbf{B}
    \end{align}
    \begin{align}
        \frac{1}{2}\sum_{i,K,L}\mathbf{A}_{K}(\mathbf{r}_{i})\cdot \mathbf{A}_{L}(\mathbf{r}_{i}) &= \frac{\alpha ^{4}}{2}\sum_{i,K,L}\frac{(\mathbf{M}_{K}\times \mathbf{r}_{iK})\cdot (\mathbf{M}_{L}\times \mathbf{r}_{iL})}{r _{iK}^{3}r _{iL}^{3}}  \nonumber \\
         &= \sum_{K,L} \mathbf{M}_{K}^{\text{T}}(\frac{\alpha ^{4}}{2}\sum_{i}\frac{\mathbf{r}_{iK}^{\text{T}}\mathbf{r}_{iL}\mathbf{I}_{3}-\mathbf{r}_{iK}\mathbf{r}_{iL}^{\text{T}}}{r _{iK}^{3}r _{iL}^{3}})\mathbf{M}_{L} \\ \nonumber
         &= \sum_{K,L} \mathbf{M}_{K}^{\text{T}}\mathbf{H}_{KL}^{\text{dia}}\mathbf{M}_{L}
    \end{align}
    Hence altogether, the diamagnetic interaction is:
    \begin{align}
        H _{\text{dia}}^{2} &= \frac{1}{2}\sum_{i}A _{i}^{2}(\mathbf{r}_{i}) \nonumber \\
         &= \mathbf{B}^{\text{T}}\mathbf{H}_{00}^{\text{dia}}\mathbf{B} + \sum_{K}\mathbf{M}_{K}^{\text{T}}\mathbf{H}_{K0}^{\text{dia}}\mathbf{B} + \sum_{K,L}\mathbf{M}_{K}^{\text{T}}\mathbf{H}_{K,L}^{\text{dia}}\mathbf{M}_{L}
    \end{align}
    with the terms:
    \begin{align}
        \mathbf{H}_{00}^{\text{dia}} & = \frac{1}{8}\sum_{i}(\mathbf{I}_{3}r _{i0}^{2} - \mathbf{r}_{i0}\mathbf{r}_{i0}^{\text{T}}) \\
        \mathbf{H}_{K0}^{\text{dia}} & = \frac{\alpha ^{2}}{2}\sum_{i}\frac{\mathbf{r}_{i0}^{\text{T}}\mathbf{r}_{iK}\mathbf{I}_{3}-\mathbf{r}_{i0}\mathbf{r}_{iK}^{\text{T}}}{r _{iK}^{3}} \\
        \mathbf{H}_{KL}^{\text{dia}} & = \frac{\alpha ^{4}}{2}\sum_{i}\frac{\mathbf{r}_{iK}^{\text{T}}\mathbf{r}_{iL}\mathbf{I}_{3}-\mathbf{r}_{iK}\mathbf{r}_{iL}^{\text{T}}}{r _{iK}^{3}r _{iL}^{3}}
    \end{align}




\end{document}