\documentclass[a4paper,11pt]{article}
\usepackage{braket}
\usepackage{physics}
\usepackage{parskip}
\usepackage{bm}
\usepackage{float}
\usepackage[utf8]{inputenc}
\usepackage{amsmath}
\usepackage{pgfplots}
\usepackage{mathrsfs}
\usepackage{simpler-wick}
\usepackage{enumerate}
\usepackage{csquotes}
\usepackage{subcaption}
\usepackage{fancyhdr}
\usepackage{tablefootnote}
\usepackage{microtype}
\usepackage{cleveref}
\usepackage{titling}
\usepackage{erewhon}
\usepackage[a4paper,width=150mm,top=25mm,bottom=25mm]{geometry}
\usepackage[Sonny]{fncychap}
\usepackage[calcwidth]{titlesec}

\usetikzlibrary{shapes.geometric, arrows}

\allowdisplaybreaks

\pgfplotsset{compat=1.7}

%fncychap layout (for chapter page)
%\renewcommand{\thechapter}{\Roman{chapter}}
\ChNameVar{\bfseries\LARGE\fontfamily{phv}}
\ChNumVar{\fontsize{50}{52}\usefont{OT1}{ptm}{m}{n}\selectfont}
\ChTitleVar{\LARGE\rm\bfseries}
\ChRuleWidth{0.8pt}

%fancyhdr layout (for header and footer)
\pagestyle{fancy}
\fancyhead{}
\fancyhead[C]{\fontsize{9}{9}\itshape{\rightmark}}
\setlength{\headheight}{15pt}

\begin{document}

\section{Hamiltonian in Magnetic Field}


    \subsection{Hamiltonian Mechanics}

    \subsection{Electromagnetic Field}

    \subsection{Hamiltonian in Electromagnetic Field}
    Schrödinger Equation:
    \begin{equation}
        i \hbar \pdv{\Psi}{t} = \frac{1}{2m} (-i\hbar \mathbf{\nabla} - z \mathbf{A})\cdot(-i \hbar \mathbf{\nabla} - z \mathbf{A})\Psi + z \phi \Psi
    \end{equation}
    which is linear on $\phi$ and quadratic on $\mathbf{A}$.

    \subsection{Spin}
    Revised one-electron Hamiltonian:
    \begin{equation}
        H = \frac{p ^{2}}{2 m _{e}} + \frac{e}{m _{e}}\mathbf{A}\cdot \mathbf{p} + \frac{e}{m _{e}}\mathbf{B}\cdot \mathbf{s} + \frac{e ^{2}}{2 m _{e}}A ^{2} + e \phi
    \end{equation}
    where the:
    \begin{itemize}
    \item $\mathbf{A}\cdot \mathbf{p}$ term is the coupling of magnetic field with electron orbital motion
    \item $\mathbf{B}\cdot \mathbf{s}$ term is the coupling of magnetic field with spin
    \item $A ^{2}$ term is the correction to the scalar potential $\phi$
    \end{itemize}

    \subsection{Molecular Electronic Hamiltonian}
    Molecular electronic Hamiltonian in atomic units:
    \begin{align}
        H = &\frac{1}{2}\sum_{i}p _{i}^{2} - \sum_{K,i}\frac{Z _{K}}{r _{iK}} + \sum_{i>j}\frac{1}{r _{ij}} + \sum_{K>L}\frac{Z _{K}Z _{L}}{R _{KL}} \\ \nonumber 
        &+ \sum_{i}\mathbf{A}_{i}(\mathbf{r}_{i})\cdot \mathbf{p}_{i} + \sum_{i}\mathbf{B}(\mathbf{r}_{i})\cdot \mathbf{s}_{i} - \sum_{i}\phi(\mathbf{r}_{i}) + \frac{1}{2}\sum_{i}A ^{2}(\mathbf{r}_{i})
    \end{align}
    In perturbative treatment we can write the Hamiltonian as:
    \begin{equation}
        H = H^{(0)} + H^{(1)} + H^{(2)}
    \end{equation}
    where
    \begin{equation}
        H^{(1)} = \sum_{i}\mathbf{A}_{i}(\mathbf{r}_{i})\cdot \mathbf{p}_{i} + \sum_{i}\mathbf{B}(\mathbf{r}_{i})\cdot \mathbf{s}_{i} - \sum_{i}\phi(\mathbf{r}_{i}) 
    \end{equation}
    and the first two terms are the paramagnetic contribution to the Hamiltonian, and the diamagnetic contribution is:
    \begin{equation}
        H^{(2)} = \frac{1}{2}\sum_{i}A ^{2}(\mathbf{r}_{i})
    \end{equation}





\end{document}