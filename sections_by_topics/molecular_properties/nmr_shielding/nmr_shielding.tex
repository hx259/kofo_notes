\documentclass[a4paper,11pt]{article}
\usepackage{braket}
\usepackage{physics}
\usepackage{parskip}
\usepackage{bm}
\usepackage{float}
\usepackage[utf8]{inputenc}
\usepackage{amsmath}
\usepackage{pgfplots}
\usepackage{mathrsfs}
\usepackage{simpler-wick}
\usepackage{enumerate}
\usepackage{csquotes}
\usepackage{subcaption}
\usepackage{fancyhdr}
\usepackage{tablefootnote}
\usepackage{microtype}
\usepackage{cleveref}
\usepackage{titling}
\usepackage{erewhon}
\usepackage[a4paper,width=150mm,top=25mm,bottom=25mm]{geometry}
\usepackage[Sonny]{fncychap}
\usepackage[calcwidth]{titlesec}

\usetikzlibrary{shapes.geometric, arrows}

\allowdisplaybreaks

\pgfplotsset{compat=1.7}

%fncychap layout (for chapter page)
%\renewcommand{\thechapter}{\Roman{chapter}}
\ChNameVar{\bfseries\LARGE\fontfamily{phv}}
\ChNumVar{\fontsize{50}{52}\usefont{OT1}{ptm}{m}{n}\selectfont}
\ChTitleVar{\LARGE\rm\bfseries}
\ChRuleWidth{0.8pt}

%fancyhdr layout (for header and footer)
\pagestyle{fancy}
\fancyhead{}
\fancyhead[C]{\fontsize{9}{9}\itshape{\rightmark}}
\setlength{\headheight}{15pt}

\begin{document}

\section{NMR Shielding}
Shielding Tensor:
\begin{equation}
    \sigma _{\beta \alpha}^{K} = \frac{\dd{^{2}E}}{\dd{B _{\alpha}}\dd{m_{K_{\beta}}}}\Big|_{{\mathbf{B},\mathbf{m}_{K} =\mathbf{0}}}
\end{equation}
How do I parameterize energy $E$ with $\mathbf{B}$ and $\mathbf{m}_{K}$? \\
The one-electronic Hamiltonian in magnetic field:
\begin{equation}
    h(\mathbf{r},\mathbf{B},\mathbf{m}) = \frac{1}{2}\bm{\pi}^{2}-\phi(\mathbf{r})
\end{equation}
in which:
\begin{equation}
    \bm{\pi} = -i \bm{\nabla} + \mathbf{A}
\end{equation}
is the kinetic momentum operator.


Vector potential:
\begin{equation}
    \mathbf{A}_{i} = \mathbf{A}_{0}(\mathbf{r}_{i}) + \sum_{K}\mathbf{A}_{K}(\mathbf{r}_{i})
\end{equation} 
with:
\begin{align}
    \mathbf{A}_{0}(\mathbf{r}_{i}) &= \frac{1}{2}\mathbf{B}\times \mathbf{r}_{0} &\mathbf{B} &= \mathbf{\nabla}\times \mathbf{A}(\mathbf{r}) \\
    \mathbf{A}_{K}(\mathbf{r}_{i}) &= \alpha ^{2}\frac{\mathbf{M}_{K}\times \mathbf{r}_{K}}{r _{K}^{3}} &\mathbf{B}_{K}(\mathbf{r}) &= \mathbf{\nabla}\times \mathbf{A}_{K}(\mathbf{r})
\end{align}
The first part is contribution from the external magnetic field, the second part from the nuclear magnetic moments.
\\
Now parameterize with MO coefficients / densities?

\newpage
\section{SCF Level}
\begin{align}
    E ^{\text{SCF}} =& \sum_{i}^{N}h _{ii} + \frac{1}{2}\sum_{ij}^{N}\langle ij||ij\rangle \nonumber \\
    =& \sum_{i}\sum_{\mu \nu}C _{\mu i} ^{*} h _{\mu \nu}C _{\nu i} + \frac{1}{2}\sum_{ij}\langle ij||ij\rangle \\
    D ^{\text{SCF}}_{\mu \nu} =& \sum_{i}C ^{*}_{\mu i}C _{\nu i} \\
    C _{\mu p}(\lambda) =& \sum_{q}C _{\mu q}(0)U _{qp}(\lambda)
\end{align}
At SCF level, the NMR shielding tensor is given as:
\begin{align}
    \sigma ^{\text{SCF},K}_{\beta \alpha} =& \frac{\dd{^{2}E ^{\text{SCF}}}}{\dd{B _{\alpha}}\dd{m _{K _{\beta}}}}\Big|_{{\mathbf{B},\mathbf{m}_{K}=0}}
\end{align}
Taking the first derivative against the nuclear magnetic moment gives:
\begin{align}
    \dv{E ^{\text{SCF}}}{m _{K _{\beta}}} =& \dv{m _{K _{\beta}}}(\sum_{i \mu \nu}C _{\mu i}^{*}h _{\mu \nu}C _{\nu i}) \nonumber \\
    =& \sum_{i \mu \nu}C _{\mu i}^{*}C _{\nu i}\dv{h _{\mu \nu}}{m _{K _{\beta}}} \nonumber \\
    =& \sum_{\mu \nu}D ^{\text{SCF}} _{\mu \nu}\dv{h _{\mu \nu}}{m _{K _{\beta}}}
\end{align}
Note that the MO coefficients are variationally determined so $\dv{\mathbf{C}}{m _{K _{\beta}}}=\mathbf{0}$, and the basis function does not depend on the nuclear magnetic moment, i.e. $\dv{\phi _{\mu}}{m _{K _{\beta}}} = 0$ . \\
Now taking the second derivative w.r.t. the external magnetic field:
\begin{align}
    \sigma _{\beta \alpha}^{\text{SCF},K} =& \frac{\dd{^{2}E ^{\text{SCF}}}}{\dd{B _{\alpha}}\dd{m _{K _{\beta}}}} \nonumber \\
    =& \dv{B _{\alpha}}(\sum_{\mu \nu}D ^{\text{SCF}} _{\mu \nu}\dv{h _{\mu \nu}}{m _{K _{\beta}}}) \nonumber \\
    =& \sum_{\mu \nu}D ^{\text{SCF}} _{\mu \nu}\frac{\dd{^{2}h _{\mu \nu}}}{\dd{B _{\alpha}}\dd{m _{K _{\beta}}}} + \sum_{\mu \nu}\dv{D ^{\text{SCF}} _{\mu \nu}}{B _{\alpha}}\dv{h _{\mu \nu}}{m _{K _{\beta}}}
\end{align}
The response of SCF density to the magnetic field perturbation is:
\begin{align}
    \dv{D _{\mu \nu}^{\text{SCF}}}{B _{\alpha}} =& \dv{B _{\alpha}}(\sum_{i}C _{\mu i}^{*}(\mathbf{B})C _{\nu i}(\mathbf{B})) \nonumber \\
    =& \dv{B _{\alpha}}(\sum_{ipq}C ^{*}_{\mu p}(0)U ^{*}_{pi}(\mathbf{B})C _{\nu q}(0)U _{qi}(\mathbf{B})) \nonumber \\
    =& \sum_{ip}C ^{*}_{\mu p}(0)\dv{U ^{*} _{pi}(\mathbf{B})}{B _{\alpha}}C _{\nu i}(\mathbf{B}) + \sum_{iq}C _{\mu i}^{*}(\mathbf{B})\dv{U _{qi}(\mathbf{B})}{B _{\alpha}}C _{\nu q}(0) \\
    =& \sum_{ip}C ^{*}_{\mu p}(U _{pi}^{B _{\alpha}})^{*}C _{\nu i} + \sum_{ip}C _{\mu i}^{*}U _{pi}^{B _{\alpha}}C _{\nu p}
\end{align}
The virtual-occupied block of $\mathbf{U}^{\mathbf{B}}$ is obtained from the CPSCF equations, and the occupied-occupied block is chosen according to the orthonormality condition.


\end{document}