\documentclass[a4paper,11pt]{article}
\usepackage{braket}
\usepackage{physics}
\usepackage{parskip}
\usepackage{bm}
\usepackage{float}
\usepackage[utf8]{inputenc}
\usepackage{amsmath}
\usepackage{pgfplots}
\usepackage{mathrsfs}
\usepackage{simpler-wick}
\usepackage{enumerate}
\usepackage{csquotes}
\usepackage{subcaption}
\usepackage{fancyhdr}
\usepackage{tablefootnote}
\usepackage{microtype}
\usepackage{cleveref}
\usepackage{titling}
\usepackage{erewhon}
\usepackage[a4paper,width=150mm,top=25mm,bottom=25mm]{geometry}
\usepackage[Sonny]{fncychap}
\usepackage[calcwidth]{titlesec}

\usetikzlibrary{shapes.geometric, arrows}

\allowdisplaybreaks

\pgfplotsset{compat=1.7}

%fncychap layout (for chapter page)
%\renewcommand{\thechapter}{\Roman{chapter}}
\ChNameVar{\bfseries\LARGE\fontfamily{phv}}
\ChNumVar{\fontsize{50}{52}\usefont{OT1}{ptm}{m}{n}\selectfont}
\ChTitleVar{\LARGE\rm\bfseries}
\ChRuleWidth{0.8pt}

%fancyhdr layout (for header and footer)
\pagestyle{fancy}
\fancyhead{}
\fancyhead[C]{\fontsize{9}{9}\itshape{\rightmark}}
\setlength{\headheight}{15pt}

\begin{document}

\section{Introduction}


    \subsection{Perturbation}

    \subsection{Electric Properties}
    In electric field $\mathbf{F}$ :
    \begin{equation}
        E(\mathbf{F}) = E _{0} - \mathbf{F}^{\text{T}}\mathbf{d}_{0} - \frac{1}{2}\mathbf{F}^{\text{T}}\mathbf{\alpha}\mathbf{F} + \order*{\mathbf{F}^{3}}
    \end{equation}
    where the molecular properties: permanent dipole moment $\mathbf{d}_{0}$ and polarisability tensor $\mathbf{\alpha}$ are:
    \begin{align}
        \mathbf{d}_{0} =& -\frac{\dd{E}}{\dd{\mathbf{F}}}\Big| _{\mathbf{0}} \\
        \mathbf{\alpha} =& -\frac{\dd{^{2}E}}{\dd{\mathbf{F}^{2}}}\Big|_{{\mathbf{0}}}
    \end{align}

    \subsection{Magnetic Properties (Closed-Shell)}
    For an NMR experiment on closed-shell systems, we have:
    \begin{itemize}
    \item Closed-shell system
    \item Static external magnetic field $\mathbf{B}$ 
    \item a set of nuclear magnetic moments $\mathbf{M}_{K}$
    \end{itemize}
    With these conditions:
    \begin{equation}
        E(\mathbf{B}, \mathbf{M}_{K}) = E _{0} + \frac{1}{2}\mathbf{B}^{\text{T}}\mathbf{E}^{(20)}\mathbf{B} + \sum_{K}\mathbf{B}^{\text{T}}\mathbf{E}_{K}^{(11)}\mathbf{M}_{K} + \sum_{K>L}\mathbf{M}_{K}^{\text{T}}\mathbf{E}_{KL}^{(02)}\mathbf{M}_{L}
    \end{equation}
    No first-order term since it vanishes for closed-shell systems (explained in section 4), and the higher order properties are neglected. The properties:
    \begin{itemize}
    \item magnetisability tensor (direct interaction of the system with external field $\mathbf{B}$):
        \begin{equation}
            \mathbf{E}^{(20)} = \mathbf{\xi} = -\frac{\dd{^{2}E}}{\dd{\mathbf{B}^{2}}}\Big|_{{\mathbf{0}}}
        \end{equation}
    \item coupling of magnetic moment $\mathbf{M}_{K}$ to field $\mathbf{B}$:
        \begin{equation}
            \mathbf{E}_{K}^{(11)} = \frac{\dd{^{2}E}}{\dd{\mathbf{M}_{K}}\dd{\mathbf{B}}} = -\mathbf{I}_{3} + \mathbf{\sigma}_{K}
        \end{equation}
        where $\mathbf{\sigma}_{K}$ is the nuclear magnetic shielding tensor
    \item coupling between nuclear magnetic moments:
        \begin{equation}
            \mathbf{E}_{KL}^{(02)}= \frac{\dd{^{2}E}}{\dd{\mathbf{M}_{K}}\dd{\mathbf{M}_{L}}}\Big|_{{0}} = \frac{\mu _{0}}{4 \pi}\frac{R ^{2}_{KL}\mathbf{I}_{3}-3 \mathbf{R}_{KL}\mathbf{R}_{KL}^{\text{T}}}{R _{KL}^{5}} + \mathbf{K}_{KL}
        \end{equation}
        the first term is the classical direct dipole-dipole interaction between $\mathbf{M}_{K}$ and $\mathbf{M}_{L}$ and the second term is the indirect nuclear spin-spin coupling tensor
    \end{itemize}



\end{document}