\documentclass[a4paper,11pt]{article}
\usepackage{braket}
\usepackage{physics}
\usepackage{parskip}
\usepackage{bm}
\usepackage{float}
\usepackage[utf8]{inputenc}
\usepackage{amsmath}
\usepackage{pgfplots}
\usepackage{mathrsfs}
\usepackage{simpler-wick}
\usepackage{enumerate}
\usepackage{csquotes}
\usepackage{subcaption}
\usepackage{fancyhdr}
\usepackage{tablefootnote}
\usepackage{microtype}
\usepackage{cleveref}
\usepackage{titling}
\usepackage{erewhon}
\usepackage[a4paper,width=150mm,top=25mm,bottom=25mm]{geometry}
\usepackage[Sonny]{fncychap}
\usepackage[calcwidth]{titlesec}

\usetikzlibrary{shapes.geometric, arrows}

\allowdisplaybreaks

\pgfplotsset{compat=1.7}

%fncychap layout (for chapter page)
%\renewcommand{\thechapter}{\Roman{chapter}}
\ChNameVar{\bfseries\LARGE\fontfamily{phv}}
\ChNumVar{\fontsize{50}{52}\usefont{OT1}{ptm}{m}{n}\selectfont}
\ChTitleVar{\LARGE\rm\bfseries}
\ChRuleWidth{0.8pt}

%fancyhdr layout (for header and footer)
\pagestyle{fancy}
\fancyhead{}
\fancyhead[C]{\fontsize{9}{9}\itshape{\rightmark}}
\setlength{\headheight}{15pt}

\begin{document}


\section{Derivative Evaluation}
    Here we explore the evaluation of the NMR shielding tensor $\mathbf{\sigma}^{K}$ of nucleus $K$ as an example of calculating the second order response properties.  
    \\
    The expression for the shielding tensor is given by:
    \begin{equation}
        \sigma ^{K}_{\alpha \beta} = \frac{\dd{^{2}E}}{\dd{B}_{\beta}\dd{m _{K \alpha}}}\Big|_{{\mathbf{B},\bm{m}_{K}=\mathbf{0}}}
    \end{equation}

    The one-electron Hamiltonian in the magnetic field is (in Atomic Unit):
    \begin{equation}
        h(\mathbf{r},\mathbf{B},\mathbf{m}) = \frac{\bm{\pi}^{2}}{2} - \phi(\mathbf{r})
    \end{equation}
    in which:
    \begin{align}
        \bm{\pi} =& \mathbf{p} + \mathbf{A}(\mathbf{r}, \mathbf{B}, \mathbf{m}) \nonumber \\
        =& -i \bm{\nabla} + \mathbf{A}(\mathbf{r}, \mathbf{B}, \mathbf{m}) \\
        \mathbf{A}(\mathbf{r},\mathbf{B},\mathbf{m}) =& \mathbf{A}_{0} + \sum_{K}\mathbf{A}_{K} \nonumber \\
        =& \frac{1}{2}\mathbf{B}\times \mathbf{r}_{0} + \alpha ^{2}\sum_{K}\frac{\mathbf{m}_{K}\times \mathbf{r}_{K}}{r _{K}^{3}}
    \end{align}
    The one-electron Hamiltonian expands to be \textcolor{cyan}{(details to be added later)}:
    \begin{equation}
        h(\mathbf{r},\mathbf{B},\mathbf{m}) = \frac{\mathbf{p} ^{2}}{2} + \mathbf{A}\cdot \mathbf{p} + \mathbf{B}\cdot \mathbf{s} + \frac{\mathbf{A}^{2}}{2} - \phi(\mathbf{r}) 
    \end{equation}
    And the molecular Hamiltonian is given by:
    \begin{equation}
        H = \sum_{i}\frac{\mathbf{p}_{i}^{2}}{2} -\sum_{k,i}\frac{Z _{K}}{r _{iK}} + \sum_{i>j}\frac{1}{r _{ij}} + \sum_{K>L}\frac{Z _{K}Z _{L}}{R _{KL}} + \sum_{i}\mathbf{A}(\mathbf{r}_{i})\cdot \mathbf{p}_{i} + \sum_{i}\mathbf{B}(\mathbf{r}_{i})\cdot \mathbf{s}_{i} - \sum_{i}\phi(\mathbf{r}_{i}) + \sum_{i}\frac{\mathbf{A}^{2}(\mathbf{r}_{i})}{2}
    \end{equation}
    \\

    To evaluate the shielding tensor, we could write it into density matrix formalism. From previous chapter, we know that, for general one-body operator $\hat{O}$ and two-body operator $\hat{G}$:
    \begin{align}
        \bar{O} =& \frac{\langle \Psi|\hat{O}|\Psi\rangle}{\langle \Psi|\Psi\rangle} = \sum_{pq}o _{pq}\gamma _{qp} \\
        \bar{G} =& \frac{\langle \Psi|\hat{G}|\Psi\rangle}{\langle \Psi|\Psi\rangle} = \frac{1}{4}\sum_{pqrs}\langle pq|\hat{g}|rs\rangle _{\text{A}} \Gamma _{rspq}
    \end{align} 
    \textcolor{magenta}{TODO: derive the expression of shielding tensor in density matrix formalism}
    \\

    \textcolor{blue}{details to be added:}
    \begin{align}
        \pdv{h}{B _{i}} =& -\frac{i}{2}(\mathbf{r}\times \bm{\nabla})_{i} \\
        \pdv{h}{m _{K j}}=& -\frac{(\mathbf{r}_{K}\times \bm{\nabla})_{j}}{\mathbf{r} _{K}^{3}}\\
        \frac{\partial ^{2}h}{\partial B _{i}\partial m _{K j}} =& \frac{1}{2}\frac{\mathbf{r}\cdot \mathbf{r}_{K}\delta _{ij}-r _{j}(\mathbf{r}_{K})_{i}}{\mathbf{r}_{K} ^{3}}
    \end{align}

    \subsection{First Derivative}
    \textcolor{red}{Taking first derivative w.r.t $m _{K}$ or $\mathbf{B}$?}\\
    Could try both way.
    \begin{align}
        \frac{\dd{\mathcal{L}_{\text{CC}}}}{\dd{m _{K \alpha}}} =& \pdv{\mathcal{L}_{\text{CC}}}{m _{K \alpha}} + \sum_{bj}\pdv{\mathcal{L}_{\text{CC}}}{\kappa _{bj}}\pdv{\kappa _{bj}}{m _{K \alpha}} + \sum_{\mu}\pdv{\mathcal{L} _{\text{CC}}}{t _{\mu}}\pdv{t _{\mu}}{m _{K \alpha}}\nonumber \\ 
        &+\sum_{\nu}\pdv{\mathcal{L}_{\text{CC}}}{\lambda _{\nu}}\pdv{\lambda _{\nu}}{m _{K \alpha}} +\sum_{z _{ai}}\pdv{\mathcal{L}_{\text{CC}}}{z _{ai}}\pdv{z _{ai}}{m _{K \alpha}} + \sum_{pq}\pdv{\mathcal{L}_{\text{CC}}}{I _{pq}}\pdv{I _{pq}}{m _{K \alpha}} \nonumber \\
        =& \pdv{\mathcal{L}_{\text{CC}}}{m _{K \alpha}} \\
        =& \text{\textcolor{blue}{density expression?}}
    \end{align}

    \subsection{Second Derivative}
    The second derivative would involve density response terms. ($D _{\mu \nu}^{B _{\beta}}$ for example)



\end{document}