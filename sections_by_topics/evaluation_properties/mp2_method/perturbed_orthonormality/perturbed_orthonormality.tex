\documentclass[a4paper,11pt]{article}
\usepackage{braket}
\usepackage{physics}
\usepackage{parskip}
\usepackage{bm}
\usepackage{float}
\usepackage[utf8]{inputenc}
\usepackage{amsmath}
\usepackage{pgfplots}
\usepackage{mathrsfs}
\usepackage{simpler-wick}
\usepackage{enumerate}
\usepackage{csquotes}
\usepackage{subcaption}
\usepackage{fancyhdr}
\usepackage{tablefootnote}
\usepackage{microtype}
\usepackage{cleveref}
\usepackage{titling}
\usepackage{erewhon}
\usepackage[a4paper,width=150mm,top=25mm,bottom=25mm]{geometry}
\usepackage[Sonny]{fncychap}
\usepackage[calcwidth]{titlesec}
\usepackage[makeroom]{cancel}

\usetikzlibrary{shapes.geometric, arrows}

\allowdisplaybreaks

\pgfplotsset{compat=1.7}

%fncychap layout (for chapter page)
%\renewcommand{\thechapter}{\Roman{chapter}}
\ChNameVar{\bfseries\LARGE\fontfamily{phv}}
\ChNumVar{\fontsize{50}{52}\usefont{OT1}{ptm}{m}{n}\selectfont}
\ChTitleVar{\LARGE\rm\bfseries}
\ChRuleWidth{0.8pt}

%fancyhdr layout (for header and footer)
\pagestyle{fancy}
\fancyhead{}
\fancyhead[C]{\fontsize{9}{9}\itshape{\rightmark}}
\setlength{\headheight}{15pt}

\begin{document}
\section{Perturbed Orthonormality Condition}
\subsection{Matrix Parameterization}
We have the general orthonormality condition, subject to perturbation, as:
\begin{equation}
    S _{pq}=\langle p|q\rangle = \delta _{pq}
\end{equation}
\begin{equation}
    \sum_{\mu \nu}C _{\mu p}^{*}S _{\mu \nu}C _{\nu q} = \delta _{pq}
\end{equation}
Parameterization of the MO coefficients:
\begin{equation}
    \mathbf{C}(\lambda) = \mathbf{C}(0)\mathbf{U}(\lambda)
\end{equation}
\begin{equation}
    C _{\mu p}(\lambda) = \sum_{r}C _{\mu r}(0)U _{rp}(\lambda)
\end{equation}
in which $\mathbf{U}(\lambda)$ is the solution to the CPHF equations.
\\

Now the orthonormality condition using this parameterization:
\begin{equation}
    \sum_{\mu \nu}\Big(\sum_{r}U _{rp}^{*}(\lambda)C _{\mu r}^{*}(0)\Big)S _{\mu \nu}(\lambda)\Big(\sum_{s}C _{\nu s}(0)U _{sq}(\lambda)\Big) = \delta _{pq}
\end{equation}
Introducing the transformed overlap matrix:
\begin{equation}
    \mathcal{S} _{pq}(\lambda) = \sum_{\mu \nu}C _{\mu p}^{*}(0)S _{\mu \nu}(\lambda)C _{\nu q}(0)
\end{equation}
we have:
\begin{equation}
    \sum_{rs}U _{rp}^{*}(\lambda)\mathcal{S}_{rs}(\lambda)U _{sq}(\lambda) = \delta _{pq}
\end{equation}
differentiating both sides of the equation gives:
\begin{equation}
    \sum_{rs}\frac{\dd{U _{rp}^{*}(\lambda)}}{\dd{\lambda}}\mathcal{S}_{rs}(\lambda)U _{sq}(\lambda) + \sum_{rs}U _{rp}^{*}(\lambda)\frac{\dd{\mathcal{S}_{rs}(\lambda)}}{\dd{\lambda}}U _{sq}(\lambda) + \sum_{rs}U _{rp}^{*}(\lambda)\mathcal{S}_{rs}(\lambda)\frac{\dd{U _{sq}(\lambda)}}{\dd{\lambda}} = 0
\end{equation}
Noting that $\bm{\mathcal{S}}(0)=\mathbf{I}$ because the unperturbed spin-orbitals are orthonormal, and it is trivial that $\mathbf{U}(0)=\mathbf{I}$. \\
Therefore evaluating the derivative at $\lambda = 0$, and denoting $A ^{\lambda} =\big(\frac{\dd{A}}{\dd{\lambda}}\big)\big|_{{\lambda=0}}$ results in:
\begin{align}
    \sum_{rs}(U _{rp}^{\lambda})^{*}\delta _{rs}\delta _{sq} + \sum_{rs}\delta _{rp}\mathcal{S}_{rs}^{\lambda}\delta _{sq} + \sum_{rs}\delta _{rp}\delta _{rs}U _{sq}^{\lambda} = 0
\end{align}
contracting the Kronecker delta tensors we get the perturbed orthonormality condition:
\begin{equation}
    (U _{qp}^{\lambda})^{*} + \mathcal{S}_{pq}^{\lambda} + U _{pq}^{\lambda} = 0
\end{equation}


\newpage
\subsection{Exponential Parameterization}
\begin{align}
    \mathbf{C}(\lambda) =& \mathbf{C}(0)\mathbf{U}(\lambda) \\
    \mathbf{U}(\lambda) =& \bm{\mathcal{S}}^{-\frac{1}{2}}(\lambda)\exp[-\bm{\kappa}(\lambda)] \\
    \bm{\mathcal{S}}(\lambda) =& \mathbf{C}^{\dagger}(0)\mathbf{S}^{\text{AO}}(\lambda)\mathbf{C}(0)
\end{align}

In this way:
\begin{align}
     \mathbf{S}(\lambda) =& \mathbf{C}^{\dagger}(\lambda)\mathbf{S}^{\text{AO}}(\lambda)\mathbf{C}(\lambda) \nonumber \\
    =& \mathbf{U}^{\dagger}(\lambda)\mathbf{C}^{\dagger}(0)\mathbf{S}^{\text{AO}}(\lambda)\mathbf{C}(0)\mathbf{U}(\lambda) \nonumber \\
    =& \mathbf{U}^{\dagger}(\lambda)\bm{\mathcal{S}}(\lambda)\mathbf{U}(\lambda) \nonumber \\
    =& \exp[-\bm{\kappa}(\lambda)]^{\dagger}\bm{\mathcal{S}}^{-\frac{1}{2} \dagger}(\lambda)\bm{\mathcal{S}}(\lambda)\bm{\mathcal{S}}^{-\frac{1}{2}}(\lambda)\exp[-\bm{\kappa}(\lambda)] \nonumber \\
    =& \exp[\bm{\kappa}(\lambda)]\exp[-\bm{\kappa}(\lambda)] \nonumber \\
    =& \mathbf{I}
\end{align}
the orthonormality is ensured by $\bm{\mathcal{S}}^{-\frac{1}{2}\dagger}\bm{\mathcal{S}}\bm{\mathcal{S}}^{-\frac{1}{2}} = \mathbf{I}$, as $\mathbf{S}^{\text{AO}}$ thus $\bm{\mathcal{S}}$ matrix is Hermitian. The perturbed orthonormality condition is then trivial:
\begin{equation}
    \mathbf{S}^{\lambda} = \pdv{\mathbf{S}(\lambda)}{\lambda}\Big|_{{\lambda = 0}} = \mathbf{0}
\end{equation}
With this new parameterization, the orthonormality condition and perturbed orthonormality condition come naturally, hence do not need to be included explicitly in the Lagrangian.\\

The derivative of $\bm{\mathcal{S}}^{-\frac{1}{2}}(\lambda)$ (needed for $\mathbf{U}^{\lambda}$ appearing in second derivative) could be solved by taking derivative on this equation:
\begin{equation}
    \bm{\mathcal{S}}^{-\frac{1}{2}\dagger}\bm{\mathcal{S}}\bm{\mathcal{S}}^{-\frac{1}{2}} = \mathbf{I}
\end{equation}
i.e. (note that $\bm{\mathcal{S}}^{\dagger} = \bm{\mathcal{S}}$ and $\bm{\mathcal{S}}(\lambda=0)=\mathbf{I}$ ):
\begin{align}
    \mathbf{0} =& \pdv{\bm{\mathcal{S}} ^{-\frac{1}{2}}}{\lambda}\bm{\mathcal{S}}\bm{\mathcal{S}}^{-\frac{1}{2}}\Big|_{{\lambda=0}} + \bm{\mathcal{S}}^{-\frac{1}{2}}\pdv{\bm{\mathcal{S}}}{\lambda}\bm{\mathcal{S}}^{-\frac{1}{2}}\Big|_{{\lambda=0}} + \bm{\mathcal{S}}^{-\frac{1}{2}}\bm{\mathcal{S}}\pdv{\bm{\mathcal{S}}^{-\frac{1}{2}}}{\lambda}\Big|_{{\lambda=0}} \nonumber \\
    =& 2 \left(\bm{\mathcal{S}}^{-\frac{1}{2}}\right)^{\lambda} + \bm{\mathcal{S}}^{\lambda}
\end{align}
\begin{equation}
    \Leftrightarrow \left(\bm{\mathcal{S}}^{-\frac{1}{2}}\right)^{\lambda} = -\frac{1}{2}\bm{\mathcal{S}}^{\lambda}
\end{equation}


\end{document}