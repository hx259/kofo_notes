\documentclass[xcolor=dvipsnames,compress,9pt]{beamer}
\usepackage[utf8]{inputenc}
\usepackage{braket}
\usepackage{physics}
\usepackage{parskip}
\usepackage{bm}
\usepackage{amsmath}
\usepackage{pgfplots}
\usepackage{mathrsfs}
\usepackage{simpler-wick}
\usepackage{caption}
\usepackage{multicol}

\pgfplotsset{compat=1.7}

\renewcommand{\indent}{\hspace*{2em}}

\usetheme{Madrid}
\useoutertheme[subsection=false]{miniframes}
\useinnertheme{circles}
\usefonttheme[onlymath]{serif}

%color theme from latexcolor.com:

%\definecolor{cherryblossompink}{rgb}{1.0, 0.72, 0.77}
%\usecolortheme[named=cherryblossompink]{structure}
%\definecolor{etonblue}{rgb}{0.59, 0.78, 0.64}
%\usecolortheme[named=etonblue]{structure}
%\definecolor{unitednationsblue}{rgb}{0.36, 0.57, 0.9}
%\usecolortheme[named=unitednationsblue]{structure}
\definecolor{tiffanyblue}{rgb}{0.04, 0.73, 0.71}
\usecolortheme[named=tiffanyblue]{structure}
%\definecolor{tearose(orange)}{rgb}{0.97, 0.51, 0.47}
%\usecolortheme[named=tearose(orange)]{structure}
%\definecolor{lightseagreen}{rgb}{0.13, 0.7, 0.67}
%\usecolortheme[named=lightseagreen]{structure}
%\definecolor{palecerulean}{rgb}{0.61, 0.77, 0.89}
%\usecolortheme[named=palecerulean]{structure}
%\definecolor{ruddypink}{rgb}{0.88, 0.56, 0.59}
%\usecolortheme[named=ruddypink]{structure}
%\definecolor{burntorange}{rgb}{0.8, 0.33, 0.0}
%\usecolortheme[named=burntorange]{structure}


\title[Molecular Properties]{Theoretical Calculation Towards Molecular Properties}
\author[H.Xu]{Hang Xu}
\titlegraphic{\includegraphics[width=5.5cm]{logo2.jpg}
    %\begin{picture}(0,0)
    %    \put(170,195){\makebox(0,0)[rt]{\includegraphics[width=3cm]{logo.png}}}
    %\end{picture}
}
\institute[Neese Group]
{
  Neese Department\\
  Max-Planck-Institut für Kohlenforschung
}
\date{\today}



%===================================================================
\setbeamertemplate{title page}
{
  \vbox{}
  \begingroup
    \begin{flushright}
        {\usebeamercolor[fg]{titlegraphic}\inserttitlegraphic\par}\vskip3em
    \end{flushright}
    \begin{beamercolorbox}[sep=8pt,center]{title}
      \usebeamerfont{title}\inserttitle\par%
      \ifx\insertsubtitle\@empty%
      \else%
        \vskip0.25em%
        {\usebeamerfont{subtitle}\usebeamercolor[fg]{subtitle}\insertsubtitle\par}%
      \fi%
    \end{beamercolorbox}%
    \vskip1em\par
    \begin{beamercolorbox}[sep=8pt,center]{author}
      \usebeamerfont{author}\insertauthor
    \end{beamercolorbox}
    \begin{beamercolorbox}[sep=8pt,center]{institute}
      \usebeamerfont{institute}\insertinstitute
    \end{beamercolorbox}
    \begin{beamercolorbox}[sep=8pt,center]{date}
      \usebeamerfont{date}\insertdate
    \end{beamercolorbox}
  \endgroup
  \vfill
}
%===================================================================

\begin{document}
\maketitle


\begin{frame}
    \frametitle{Molecular Properties as Energy Derivatives}
    Hamiltonian subject to some external perturbation:
    \begin{equation}
        \hat{H} = \hat{H}(0)+\lambda \hat{O}+\dots
    \end{equation}
    Example of external perturbations:
    \begin{multicols}{2}
        \begin{itemize}
            \item $\mathbf{F}$: external electric field
            \item $\mathbf{B}$: external magnetic field
            \item $\mathbf{I}$: nuclear spin
            \item $\mathbf{R}$: nuclear geometric displacement
        \end{itemize}
    \end{multicols}
    Energy expansion in terms of external perturbation:
    \begin{equation}
        E(\lambda) = E(0) + \lambda \frac{\dd{E}}{\dd{\lambda}}\Big|_{{\lambda=0}} + \frac{1}{2!}\lambda ^{2}\frac{\dd{^{2}E}}{\dd{\lambda ^{2}}}\Big|_{{\lambda=0}} + \dots
    \end{equation}

    The n-th order property could be written as the n-th order derivative against perturbation.

\end{frame}

\begin{frame}
    \frametitle{Types of Molecular Properties}
    Some first and second order molecular properties, expressed as energy derivatives: 
    \begin{equation}
        \frac{\partial ^{n _{1}+n _{2}}E}{\partial ^{n _{1}}\lambda _{1}\partial ^{n _{2}}\lambda _{2}} \nonumber
    \end{equation}
    \begin{center}
       \begin{tabular}{ c  c  c  c  l}
           \hline
           $n _{\mathbf{F}}$ & $n _{\mathbf{B}}$  & $n _{\mathbf{I}}$ & $n _{\mathbf{R}}$ & Property \\
           \hline
           1 & 0 & 0 & 0 & electric dipole moment \\
           0 & 1 & 0 & 0 & magnetic dipole moment \\
           0 & 0 & 1 & 0 & hyperfine coupling constant \\
           0 & 0 & 0 & 1 & nuclear gradient \\
           \hline
           2 & 0 & 0 & 0 & electric polarizability \\
           0 & 2 & 0 & 0 & magnetizability \\
           0 & 0 & 2 & 0 & nuclear spin-spin coupling \\
           0 & 0 & 0 & 2 & harmonic vibrational frequencies \\
           1 & 0 & 0 & 1 & infrared absorption intensities \\
           0 & 1 & 1 & 0 & NMR shielding \\
           \hline
           \footnote{Introduction to computational chemistry - Jensen, 2007}
       \end{tabular} 
    \end{center}
\end{frame}

\begin{frame}
   \frametitle{Property Evaluation: Example - NMR Shielding Tensor} 
    NMR Shielding Tensor:
    \begin{align}
        \sigma _{\beta \alpha}^{K} =& \frac{\dd{^{2}E}}{\dd{B _{\alpha}}\dd{m_{K_{\beta}}}}\Big|_{{\mathbf{B},\mathbf{m}_{K} =\mathbf{0}}} \nonumber \\
        =& \sum_{\mu \nu}D _{\mu \nu}\frac{\dd{^{2}h _{\mu \nu}}}{\dd{B _{\alpha}}\dd{m _{K _{\beta}}}} + \sum_{\mu \nu}\dv{D _{\mu \nu}}{B _{\alpha}}\dv{h _{\mu \nu}}{m _{K _{\beta}}}
    \end{align}
    $D _{\mu \nu}$ is the density matrix specific to different levels of theory, e.g. $D ^{\text{SCF}}$, $D ^{\text{MP2}}$, $D ^{\text{CC}}$ \dots   
    \\~\\
    \pause
    Energy functionals:
    \begin{align}
        \mathcal{L}_{\text{MP2}} =& E _{\text{HF}} + E _{\text{H}} + \sum_{ai}f _{ai}z _{ai} \\
        \mathcal{L}_{\text{CC}} =& \langle \Phi _{0}|(1 + \hat{\Lambda})\mathcal{H}|\Phi _{0}\rangle + \sum_{ai}f _{ai}z _{ai}
    \end{align}
    Corresponding (relaxed) density:
    \begin{align}
        D ^{\text{R}}_{\mu \nu} =& D ^{\text{amp}} _{\mu \nu} + D ^{z}_{\mu \nu}
    \end{align}
\end{frame}

\begin{frame}
    \frametitle{Current Progress}
    Have done:
    \begin{itemize}
    \item Derivation: MP2 level equations derived by hand, CC equations in progress
    \item Implementation: hand-written code for MP2 UHF energy and relaxed density now in ORCA-AUTOCI module
    \end{itemize}
    \pause
    TODO:
    \begin{itemize}
    \item To connect the MP2 code with SHARK to get density derivative hence the second-order properties like NMR shielding
    \item Maybe use ORCA-AGE to generate automatic code for comparison
    \end{itemize}
    \pause
    Further Plan:
    \begin{itemize}
    \item Implement Coupled Cluster derivatives
    \item Local approximation: DLPNO
    \end{itemize}
    \pause
    \begin{align}
    \text{Acknowlegement: }&\text{Prof. Dr. Frank Neese} \nonumber \\
                    &\text{Stan Papadopoulos}\nonumber \\
                    &\text{Dr. Georgi Stoychev}\nonumber
    \end{align}
\end{frame}
    
\end{document}