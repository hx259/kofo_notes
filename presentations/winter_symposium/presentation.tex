\documentclass[xcolor=dvipsnames,compress,9pt]{beamer}
\usepackage[utf8]{inputenc}
\usepackage{braket}
\usepackage{physics}
\usepackage{parskip}
\usepackage{bm}
\usepackage{amsmath}
\usepackage{pgfplots}
\usepackage{mathrsfs}
\usepackage{simpler-wick}
\usepackage{caption}
\usepackage{multicol}

\pgfplotsset{compat=1.7}

\renewcommand{\indent}{\hspace*{2em}}

\usetheme{Madrid}
\useoutertheme[subsection=false]{miniframes}
\useinnertheme{circles}
\usefonttheme[onlymath]{serif}

%color theme from latexcolor.com:

%\definecolor{cherryblossompink}{rgb}{1.0, 0.72, 0.77}
%\usecolortheme[named=cherryblossompink]{structure}
%\definecolor{etonblue}{rgb}{0.59, 0.78, 0.64}
%\usecolortheme[named=etonblue]{structure}
%\definecolor{unitednationsblue}{rgb}{0.36, 0.57, 0.9}
%\usecolortheme[named=unitednationsblue]{structure}
\definecolor{tiffanyblue}{rgb}{0.04, 0.73, 0.71}
\usecolortheme[named=tiffanyblue]{structure}
%\definecolor{tearose(orange)}{rgb}{0.97, 0.51, 0.47}
%\usecolortheme[named=tearose(orange)]{structure}
%\definecolor{lightseagreen}{rgb}{0.13, 0.7, 0.67}
%\usecolortheme[named=lightseagreen]{structure}
%\definecolor{palecerulean}{rgb}{0.61, 0.77, 0.89}
%\usecolortheme[named=palecerulean]{structure}
%\definecolor{ruddypink}{rgb}{0.88, 0.56, 0.59}
%\usecolortheme[named=ruddypink]{structure}
%\definecolor{burntorange}{rgb}{0.8, 0.33, 0.0}
%\usecolortheme[named=burntorange]{structure}


\title[Molecular Properties \&  Coupled Cluster]{Introduction: Evaluation of Electric and Magnetic Molecular Properties using Canonical Coupled Cluster Method}
\author[H.Xu]{Hang Xu}
\institute[MPI-KoFo]
{
  Neese Department\\
  Max-Planck-Institut für Kohlenforschung
}
\date{\today}

\begin{document}
\maketitle

\begin{frame}
    \frametitle{Second-Order Electric and Magnetic Properties}
    Hamiltonian subject to perturbation:
    \begin{equation}
        \hat{H} = \hat{H}(0)+\lambda \hat{O}
    \end{equation}
    E.g. $\hat{O} = -\bm{\mathcal{E}}\cdot \sum_{\mu}q _{\mu}\mathbf{r}_{\mu}$ for external electric field. \\
    \pause

    Energy response subject to external perturbation (Taylor series):
    \begin{equation}
        E(\lambda) = E(0) + \lambda \frac{\dd{E}}{\dd{\lambda}}\Big|_{{\lambda=0}} + \frac{1}{2!}\lambda ^{2}\frac{\dd{^{2}E}}{\dd{\lambda ^{2}}}\Big|_{{\lambda=0}} + \dots
    \end{equation}
    \pause 

    By definition of expectation value and Perturbation Theory:
    \begin{equation}
        \langle \hat{O}\rangle = \langle \hat{H}^{(1)}\rangle = \frac{\langle \Psi ^{(0)}|\hat{H}^{(1)}|\Psi ^{(0)}\rangle}{\langle \Psi ^{(0)}|\Psi ^{(0)}\rangle} = \frac{\dd{E(\lambda)}}{\dd{\lambda}}\Big|_{{\lambda=0}}
    \end{equation}
    \pause

    Therefore we can evaluate many molecular properties with energy derivatives. For example:

    \begin{multicols}{2}
    \begin{itemize}
    \item $\frac{\dd{^2 E}}{\dd{\mathcal{E} _{\alpha}}\dd{\mathcal{E} _{\beta}}}$: electric polarizability
    \item $\frac{\dd{^2 E}}{\dd{B _{\alpha}}\dd{B _{\beta}}}$: magnetizability
    \item $\frac{\dd{^2 E}}{\dd{B _{\alpha}}\dd{m _{K \beta}}}$: NMR shielding tensor
    \item $\frac{\dd{^2 E}}{\dd{B _{\alpha}}\dd{S _{\beta}}}$: electronic g-tensor
    \end{itemize}
    \end{multicols}
\end{frame}

\begin{frame}
    \frametitle{Canonical Coupled Cluster Ansatz}
    The Coupled Cluster wavefunction:
    \begin{equation}
        |\Psi _{\textrm{CC}}\rangle=e ^{\hat{T}}|\Phi _{0}\rangle
    \end{equation}
    \begin{align}
        &\hat{T} = \sum_{\eta}\hat{T}_{\eta} &\hat{T}_{\eta} = \frac{1}{(\eta !)^{2}}\sum_{\substack{ij\dots \\ ab \dots}}t _{ij \dots}^{ab \dots}\{\hat{a}^{\dagger}\hat{i}\hat{b}^{\dagger}\hat{j}\}
    \end{align}
    \pause
    Energy expression:
    \begin{equation}
       \Delta E _{\textrm{CC}} = \langle \Psi _{\textrm{CC}}|\hat{H}_{\textrm{N}}|\Psi _{\textrm{CC}}\rangle = \langle \Phi _{0}|\mathcal{H}|\Phi _{0}\rangle
    \end{equation}
    \pause
    Effective Hamiltonian (using BCH expansion):
    \begin{align}
        \mathcal{H} =& e ^{-\hat{T}}\hat{H}_{\textrm{N}}e ^{\hat{T}} \nonumber \\
        =& \hat{H}_{\textrm{N}} + [\hat{H}_{\textrm{N}},\hat{T}] + \frac{1}{2!}[[\hat{H}_{\textrm{N}},\hat{T}],\hat{T}] + \frac{1}{3!}[[[\hat{H}_{\textrm{N}},\hat{T}],\hat{T}],\hat{T}] \nonumber \\
        &+ \frac{1}{4!}[[[[\hat{H}_{\textrm{N}},\hat{T}],\hat{T}],\hat{T}],\hat{T}] \nonumber \\
        =&\hat{H}_{\textrm{N}} + (\hat{H}_{\textrm{N}}\hat{T})_{\textrm{C}} + \frac{1}{2!}(\hat{H}_{\textrm{N}}\hat{T}^{2})_{\textrm{C}} + \frac{1}{3!}(\hat{H}_{\textrm{N}}\hat{T}^{3})_{\textrm{C}} + \frac{1}{4!}(\hat{H}_{\textrm{N}}\hat{T}^{4})_{\textrm{C}} \nonumber \\
        =& (\hat{H}_{\textrm{N}}e ^{\hat{T}})_{\textrm{C}}
    \end{align}
\end{frame}

\begin{frame}
    \frametitle{CC Lagrangian and Stationary Conditions}
    For non-variational methods, Lagrange's undetermined multipliers ansatz are used.
    \begin{align}
        \mathcal{L}_{\textrm{CC}} =& \langle \Phi _{0}|\mathcal{H}|\Phi _{0}\rangle + \sum_{\eta}\lambda _{\eta}\langle \Phi _{\eta}|\mathcal{H}|\Phi _{0}\rangle
        + \sum_{ai}f _{ai}z _{ai} + \sum_{pq}I _{pq}(\sum_{\mu \nu}C ^{*}_{\mu p}S _{\mu \nu}C _{\nu q} - \delta _{pq}) \nonumber \\
         =& \langle \Phi _{0}|(1+\hat{\Lambda})\mathcal{H}|\Phi _{0}\rangle + \sum_{ai}f _{ai}z _{ai} + \sum_{pq}I _{pq}(\sum_{\mu \nu}C ^{*}_{\nu p}S _{\mu \nu}C _{\nu q} - \delta _{pq})
    \end{align}
    \pause
The following stationary conditions need to be imposed, resulting different sets of equations to solve:
\begin{itemize}
\item $\pdv{\mathcal{L}_{\textrm{CC}}}{\lambda _{\eta}}=0$: resulting the CC amplitude equations
\pause
\item $\pdv{\mathcal{L}_{\textrm{CC}}}{t _{\eta}}=0$: resulting the CC $\Lambda$ equations
\pause
\item $\pdv{\mathcal{L}_{\textrm{CC}}}{z _{ai}}=0$: resulting the Brillouin condition (HF equations)
\pause
\item $\pdv{\mathcal{L}_{\textrm{CC}}}{\kappa _{bj}}=0$: resulting the z-vector equation
\pause
\item $\pdv{\mathcal{L}_{\textrm{CC}}}{I _{pq}}=0$: resulting the orthonormality condition
\end{itemize}
\end{frame}

\begin{frame}
    \frametitle{Evaluation of Derivatives: NMR Shielding Tensor}
    Expression for the shielding tensor:
    \begin{align}
        \sigma ^{K} _{\beta \alpha} =& \Big(\frac{\dd{^{2}E}}{\dd{B _{\alpha}}\dd{m _{K \beta}}}\Big)\Big|_{{\mathbf{B},\mathbf{m}_{K}=\mathbf{0}}} \nonumber \\
        =& \sum_{\mu \nu}D _{\mu \nu} \frac{\partial ^2 h _{\mu \nu}}{\partial B _{\alpha}\partial m _{K \beta}} + \sum_{\mu \nu} \pdv{D _{\mu \nu}}{B _{\alpha}}\pdv{h _{\mu \nu}}{m _{K \beta}}
    \end{align}
    \pause
    One-electron Hamiltonian in magnetic field:
    \begin{equation}
        h = \frac{\mathbf{p}^{2}}{2} + \mathbf{A}\cdot \mathbf{p} + \mathbf{B}\cdot \mathbf{s} + \frac{\mathbf{A}^{2}}{2} - \phi(\mathbf{r})
    \end{equation}
    \pause
    Then the derivatives of the one-electron Hamiltonian could be worked out.\\
    Taking derivatives of the Lagrangian $\mathcal{L}_{\textrm{CC}}$ is not too difficult at this point, as the Lagrangian is already made stationary w.r.t most parameters. \\
    \pause
    Note the derivatives of MO coefficient $\mathbf{C}$ are worked out by firstly solving the CPSCF equations for the orbital transformation parameter, then for example:
    \begin{equation}
        \pdv{C _{\mu p}}{B _{\alpha}} = \sum_{q}C _{\mu q}U ^{B _{\alpha}}_{qp}
    \end{equation}

\end{frame}

\begin{frame}
    \frametitle{Long-Term Goal: Evaluation of Properties with DLPNO-CC method}
    \begin{center}
    \begin{tabular}{||c|c||}
            \hline
            Method & \footnotesize{Computational Scaling} \\
            \hline
            \hline
            CCSD & $\order*{N ^{6}}$ \\
            CCSD(T) & $\order*{N ^{7}}$ \\
            CCSDT & $\order*{N ^{8}}$ \\
            \hline
    \end{tabular}
    \end{center}
    It's not impossible, but computationally unfriendly hence often not feasible to calculate properties for large molecular systems using canonical CC ansatz.
    \\~\\
    \pause
    Taking local correlation approximations would significantly reduce the amount of calculation,
    making the methods in previous slides applicable to real-life chemistry.
    \pause
    \\~\\
    \begin{center}
    \Huge Thank you for your time!
    \end{center}
    \pause
    \begin{align}
    \text{Acknowlegement: }&\text{Prof. Dr. Frank Neese} \nonumber \\
                    &\text{Georgi Stoychev}\nonumber \\
                    &\text{Stan Papadopoulos}\nonumber 
    \end{align}
\end{frame}
    
\end{document}